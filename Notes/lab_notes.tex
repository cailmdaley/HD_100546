%%% Laboratory	 Notes
%%% Template by Mikhail Klassen, April 2013
%%% Contributions from Sarah Mount, May 2014
\documentclass[a4paper]{tufte-handout}

\newcommand{\workingDate}{\textsc{Summer $|$ 2016}}
\newcommand{\userName}{Cail Daley}
\newcommand{\institution}{Leiden University}

\usepackage{lab_notes}
\usepackage{url}
\usepackage{tabu}

\setcitestyle{numbers}

\usepackage{hyperref}
\hypersetup{
    pdffitwindow=false,            % window fit to page
    pdfstartview={Fit},            % fits width of page to window
    pdftitle={HD_100546_Modeling_Notes},     % document title
    pdfauthor={Cail Daley},         % author name
    pdfsubject={},                 % document topic(s)
    pdfnewwindow=true,             % links in new window
    colorlinks=true,               % coloured links, not boxed
    linkcolor=DarkScarletRed,      % colour of internal links
    citecolor=DarkChameleon,       % colour of links to bibliography
    filecolor=DarkPlum,            % colour of file links
    urlcolor=DarkSkyBlue           % colour of external links
}


\title{HD 100546 Modeling}
\date{2016}

\begin{document}
\maketitle

%%%%%%%%%%%%%%%%%%%%%%%%%%%%%%%%%%%%%%%%%%%%%%%%%%%%%%%%

\begin{tasks}
    \begin{itemize}
    	\item Check how aspect is defined in amom.py
      \item Add isovelocity contours to channel maps and mirror plot and add beam (make white?)
      \item Make table of best fits for different parameter sets
      \item Add flaring index and warp
    \end{itemize}
\end{tasks}

%%%%%%%%%%%%%%%%%%%%%%%%%%%%%%%%%%%%%%%%%%%%%%%%%%%%%%%%

\begin{maybe}
    \begin{itemize}
    	\item Stellar mass error bars?
      \item Reclean with different start velocity-- increase by half a channel (0.075 km/s)?
      \item Spectral profile of outer disk? Use polygon to make a ring, compare to Pineda
    \end{itemize}
\end{maybe}

%%%%%%%%%%%%%%%%%%%%%%%%%%%%%%%%%%%%%%%%%%%%%%%%%%%%%%%%

\newday{5 July 2016}

\newthought{Started taking notes on modeling work}

Over the past two weeks, I have been using the script amom.py (jointly written by Catherine Walsh and Atilla Juhasz) to model the first moment map of HD 100546. This is done with the help of a front end, exectute\_amom.py, that reads in the disk's first moment map, creates an ideal model given certain parameters, and then convolves the model with the ALMA image's beam. Current model parameter information can be found in Table \ref{tab:parameters}.

\begin{table}[h]
\label{tab:parameters}
\centering
\caption{HD 100546 Parameters}
\begin{tabular}{lrr}
\toprule
Parameter 	&	Start Value &	 Preferred Value\\
\midrule
Stellar Mass &	2.4 &	N/A\\
Distance (parsecs)&	96.9 &	N/A\\
Aspect Ratio   &	0.0   &	 0.02\\
Cone           &	Lower &	 Lower\\
Position Angle &	146   &	 144\\
Inclination    &	45   &	 36 \\
\end{tabular}
\end{table}

execute\_amom.py creates a model for each possible permutation of a set of parameters. For each model, a plot is saved containing four images:
\begin{enumerate}
  \item ALMA data first moment map
  \item Convolved model first moment map
  \item Residuals $(data-model)$
  \item Normalized Residuals $\frac{(data-model)}{data}$
\end{enumerate}

For each combination of aspect ratio and cone, a plot is saved containing the fit statistics of each model in a square matrix of position values and inclinations. Position angle currently ranges between 130 and 160, and inclination ranges between 30 and 60. Two goodness of fit quantifications are used: a $\chi^2$ with $\sigma=1$ (i.e. $\sum(data-model)^2$), and the peak residual. The sum of squares tends to be better for assessing overall goodness of fit, while the peak reveals where and how badly the model fails to fit the data.  The the models with the smallest sums of squares and smallest peaks are summarized in Tables \ref{tab:best sums} and \ref{tab:best peaks}, respectively.

\begin{table}[p]
\label{tab:best sums}
\caption{Best-Fitting Models (Sum of Squares)}
\begin{tabu} to \textwidth {X[r]X[r]X[r]X[r]X[r]}
  \toprule
  Aspect Ratio & Cone    & Position Angle & Inclination & Sum of Squares\\
  \midrule
  0            &	N/A    & 144            & 36          &	75.28          \\
  0.01         &	Lower  & 144            & 36          &	74.84          \\
  0.01         &	Upper  & 144            & 36          &	75.91          \\
  0.02         &	Lower  & 144            & 36          &	74.61          \\
  0.02         &	Upper  & 144            & 36          &	76.75          \\
  0.03         &	Lower  & 144            & 36          &	\textbf{74.58}          \\
  0.03         &	Upper  & 144            & 36          &	77.80          \\
  0.04         &	Lower  & 144            & 36          &	74.76          \\
  0.04         &	Upper  & 144            & 36          &	79.05          \\
  0.1          &	Lower  & 144            & 36          &	80.10          \\
  0.1          &	Upper  & 144            & 36          &	90.94          \\
  0.2          &	Lower  & 144            & 36          &	106.05          \\
  0.2          &	Upper  & 142            & 36          &	128.39          \\
  0.3          &	Lower  & 144            & 34          &	149.06          \\
  0.3          &	Upper  & 142            & 34          &	185.48          \\
  0.4          &	Lower  & 144            & 32          &	215.49          \\
  0.4          &	Upper  & 142            & 34          &	267.70          \\
  0.5          &	Lower  & 144            & 32          &	300.60          \\
  0.5          &	Upper  & 142            & 32          &	370.85          \\
\end{tabu}
\end{table}

\begin{table}[p]
\label{tab:best peaks}
\caption{Best-Fitting Models (Peak Residuals)}
\begin{tabu} to \textwidth {X[r]X[r]X[r]X[r]X[r]X[r]}
  \toprule
  Aspect Ratio & Cone & Position Angle & Inclination &	Peak Offset   & Peak Residual\\
  \midrule
  0            & N/A  & 130            & 30          & $(0.24, 0.12)$ &	1.31\\
  0.01         & Lower& 130            & 30          & $(0.24, 0.12)$ &	1.32\\
  0.01         & Upper& 130            & 30          & $(0.24, 0.12)$ &	1.31\\
  0.02         & Lower& 130            & 30          & $(0.24, 0.12)$ &	1.32\\
  0.02         & Upper& 130            & 30          & $(0.24, 0.12)$ &	1.31\\
  0.03         & Lower& 130            & 30          & $(0.24, 0.12)$ &	1.32\\
  0.03         & Upper& 130            & 30          & $(0.24, 0.12)$ &	1.31\\
  0.04         & Lower& 130            & 30          & $(0.24, 0.12)$ &	1.32\\
  0.04         & Upper& 130            & 30          & $(0.24, 0.12)$ &	1.31\\
  0.1          & Lower& 130            & 30          & $(0.24, 0.12)$ &	1.33\\
  0.1          & Upper& 130            & 30          & $(0.24, 0.12)$ &	1.29\\
  0.2          & Lower& 130            & 30          & $(0.24, 0)   $ &	1.36\\
  0.2          & Upper& 130            & 30          & $(0.24, 0.12)$ &	1.27\\
  0.3          & Lower& 130            & 30          & $(0.24, 0)   $ &	1.40\\
  0.3          & Upper& 130            & 30          & $(0.24, 0.12)$ &	1.25\\
  0.4          & Lower& 130            & 30          & $(0.24, 0)   $ &	1.45\\
  0.4          & Upper& 130            & 30          & $(0, -0.24)  $ &	\textbf{1.23}\\
  0.5          & Lower& 130            & 30          & $(0.24, 0)   $ &	1.49\\
  0.5          & Upper& 132            & 30          & $(0.24, 0.12)$ &	1.26\\
\end{tabu}
\end{table}

 Stellar masses I have found:\\
  2.4 (van den Ancker et al.) fits model better, but is older and the original source seems to be gone\\
  2.5 (Manoi et al.)\\
  4  (Levenhagen and Leister)

\hrulefill

%%%%%%%%%%%%%%%%%%%%%%%%%%%%%%%%%%%%%%%%%%%%%%%%%%%%%%%%

\bibliographystyle{plainnat}
\bibliography{lab_notes}

\end{document}
