
\documentclass{knac}

% This next line is only necessary to get a live hyperlink in the
% output document:
\usepackage{hyperref}


%  If you want to define shortcuts for commonly-typed commands you'll
%  use, you can do so here - for example, here is a command to make it
%  easier to refer to H alpha emission (using 'math mode', denoted by
%  the dollar signs, to generate the Greek letter alpha).

\newcommand{\ha}{H$\alpha$}


\begin{document}

% Give the title of the paper here.
\title{Modelling the Kinematics of HD 100546: ALMA Evidence for a Planetary Companion?}

% Don't use the AASTeX-style commands \affil, \email, etc. for
% specifying various parts of the author or advisor information here -
% just give the author(s) and affiliation(s) just as you would like
% them to appear in the document:

\author{Cail Daley, Wesleyan University}

% If there are multiple advisors, use \advisors instead.
\advisor{Catherine Walsh, Leiden University}


\begin{abstract}

  This document serves both as an example of how a symposium
  proceedings paper will appear, and as a guide to the procedure
  necessary to produce the paper in the proper format. Several key
  concepts are outlined, including how to cite references, examples of
  how to include figures and tables, and where to get help when you
  have questions.  To get the most out of this, you should read
  through the PDF output document produced, but also look at the
  source file (with the .tex extension), where the comments will give
  you additional information about how to produce the output you see
  here.

\end{abstract}

% After this comes your document.  You can use section headings with
% the \section command.  (And also \subsection if you wish.)

\section{Introduction}

This is where the introductory text of the paper would go.  For
example, we can cite references in the text by using the \verb|\citep|
and \verb|\citet| commands, to cite the work of, for example,
\citet{Bary2002}, and others \citep{Kwitter2013, Willman2012}.  For
including references in the final reference list, you can get
Latex-formatted references directly from NASA ADS
(\url{http://www.adsabs.harvard.edu/abstract_service.html}) - after
finding the reference you want, click on ``Preferred format for this
abstract'' and you will get a \verb|\bibitem| record in \LaTeX\ format
that can be pasted into the references section of your document, and
referred to in the paper by a shorthand tag that you create; see the
source file of this document to see how it works.

% Here is code for an example figure, illustrating how to include the
% graphics file.  Notice that it's not necessary to number the figures
% or tables explicitly - Latex takes care of that, numbering them in
% the order they appear.

% The file included in this example is a Postscript file.  (Postscript
% is a graphics format open used for scientific figures.)  It is also
% possible to include figures in PDF format if you prefer, by replacing
% the filename with the name of your desired PDF file, and by
% compiling this document with the command 'pdflatex' instead of just
% 'latex'.   If you are using TeXShop on a Mac to compile the file,
% note that by default it will call 'pdflatex' behind the scenes to
% compile your file.  That's fine if you want to use PDF figures, but
% if you want to use Postscript figures instead, then go the
% Preferences in TeXShop, choose the "Typesetting" tab, and change
% "Default Script" from "Pdftex" to "TeX + DVI".

% If you are playing around with this file to learn about Latex, try
% changing the filename below in the \includegraphics command to
% reference one of your own figures.

\begin{figure}[bht]
% Make sure the figure is centered:
\centering
% Include the file - notice that we can rotate it and scale it as
% needed:
% Give the caption for the Figure here.
\caption{Example figure: The spectral energy distribution of DoAr 21,
  from ultraviolet (4000 \AA) through infrared wavelengths. Spitzer and ISO data
  are shown as filled squares, and the 850 \micron\ and 1.3 mm
  upper limits are not shown.  The photometry has been de-reddened
  with $A_V = 6.2$ and $R_V = 4.2$ as described in the text.
  Overplotted is a model photosphere (solid line); the star shows a
  clear infrared excess at $\lambda \ge 8$ \micron.}
% This label can be used in the text to refer to the figure by number, if
% desired; see example \ref command in the text where this figure is
% referred to.
\label{figure:sed}
\end{figure}


% Here's an example table, using the AASTeX \deluxetable
% environment.  If you don't need the fancy features of \deluxetable,
% you can just use \table or \tabular.  I haven't tested it
% extensively, but it seems like using \deluxetable makes it a little
% harder to squeeze tables into a smaller amount of space (and gives
% you a little less control over their placement) than the regular
% \table environment.

\begin{deluxetable}{lccc}
\centering
% This sets the table to its 'natural' width, i.e. not trying to
% expand to the width of the text:
\tablewidth{0pt}
\tablecaption{Example table: Photometry of DoAr 21\label{table:gemini-photometry}
}
\tablehead{
  \colhead{Filter} & \colhead{Flux within  0\farcs82} &
 \colhead{Flux within 2\farcs75} &  \colhead{Flux within 6\arcsec}\\
\colhead{} & \colhead{(mJy)} &\colhead{(mJy)}&\colhead{(mJy)}}
\startdata
\phn 8.6 \micron\ (PAH)  & 313  & 525  & 596 \\
10.4 \micron\  (Si-4)      & 176  & 300  & 347 \\
11.3 \micron\ (PAH) & 234  & 680  & 816 \\
%18.3 \micron\ (Qa)       & 165  & 607  & 728 \\
\enddata
\end{deluxetable}

\section{Observations and/or Theory}

Observations could go here (Table \ref{table:gemini-photometry}), and
so on.  We might also want to include figures in your paper (Figure
\ref{figure:sed}).   Note that the symposium proceedings are printed in
black and white, so you should either convert your figures to
grayscale, or at minimum ensure that they are still legible when
printed that way.

Or perhaps your work is more theoretical, and has some mathematical
notation embedded in it.  It's easy to display math in-line (e.g.\ the
code  \verb|$L = 4 \pi R_*^2 \sigma T^4$| produces $L = 4
\pi R_*^2 \sigma T^4$) or to have equations that are separated from
the rest of the text, either using \$\$ or  \verb|\begin{equation}|
and  \verb|\end{equation}|  commands to surround them:

\begin{equation}
F_\nu = \frac{L_\nu}{4\pi d^2}
\end{equation}




\section{Length Guidelines}

You should aim for a five-page paper (or eight pages if you are
co-authoring a paper with multiple students).  However, one of the
slightly tricky things about \LaTeX\ is that you don't have much
control over where it places figures and tables. (If you look at the
source for this file, you may find that the figure and table don't
come out in the document exactly where the code to produce them was
located.  They will never come out {\em earlier} in the text, but they
will often come out later.)  If you find that \LaTeX\ is (perhaps
inexplicably) pushing your paper to six pages because it insists on
putting a table or figure onto a different page, or leaves some blank
space on a page, you can try playing around with where in the text you
insert the commands to produce the figure or table.  Ultimately, if
you have a hard time getting it work, don't drive yourself (or your
advisor) crazy trying to fiddle with it.  If you can see that the
actual content of your paper could reasonably be fit into five pages,
go ahead and submit the paper as-is, and the proceedings editor will
try to tweak it to fit it into five pages.

\section{Getting Help}

See the \LaTeX\ source code for this example document to get a sense
of how particular commands produce the output you see here.  The
source document also has extensive comments explaining what the
structure of the document is, and how particular commands work.

There are many print and on-line resources for help with \LaTeX.  See
the AAS web page \url{http://aas.org/aastex/aastex-documentation} for
help with commands specific to AASTeX.  In particular, there is a list
of AASTeX symbols, including favorites like \verb|\sun| ($\sun$) and
the Angstrom symbol \verb|\AA| (\AA), at
\url{http://authortools.aas.org/aastex/aassymbols.pdf}.

For citing references in the text and in the bibliography, most of
this will take care of itself automatically if you copy your
bibliographic entries from NASA ADS and cite them in the text using
\verb|\citet| or \verb|\citep| as outlined above.  But if you have any
additional questions (e.g., how to cite something not listed in ADS,
follow the guidelines given for the {\it Astrophysical Journal}\/ at
\url{http://aas.org/authors/manuscript-preparation-aj-apj-author-instructions}.

If you have questions about working with this \LaTeX\ style, or about
the proceedings in general, feel free to e-mail the proceedings editor
Roy Kilgard at \href{mailto:rkilgard@wesleyan.edu}{\nolinkurl{rkilgard@wesleyan.edu}}.


\section{Deadlines}

Completed papers are due by 5 PM on Friday, September 23.  Send them by
e-mail to Roy Kilgard,
\href{mailto:rkilgard@wesleyan.edu}{\nolinkurl{rkilgard@wesleyan.edu}}.
Your submission should include the source \LaTeX\ file (the text file
with the .tex extension), as well as the files for any figures that
are referenced by the .tex file.  You can send files as individual
e-mail attachments, or you can put everything together in a single zip
or tar file.


% Give the acknowledgments here.  Citing the NSF grant that funds the
% REU program is a good idea (just using the text below).  In the
% interest of saving space, there is no section heading inserted here
% specifically for acknowledgments.

\acknowledgments

Here is where you thank people (and funding agencies!)  There is no
explicit section header in the output; just make this the last thing
before the bibliography, preceded by the \verb|\acknowledgments|
command. For example:  We gratefully acknowledge the National
Science Foundation's support of the Keck Northeast Astronomy
Consortium's REU program through grant AST-1005024. This research has
made use of the SIMBAD database, operated at CDS, Strasbourg, France,
and of NASA's Astrophysics Data System.



% Here's how we specify the bibliography.

\begin{thebibliography}{}

% For including references in the final reference list, you can get
% Latex-formatted references directly from NASA ADS
% (http://www.adsabs.harvard.edu/abstract_service.html) - after
% finding the reference you want, click on "Preferred format for this
% abstract" and you will get a \bibitem record in Latex format that
% can be pasted in your document.

% Note that each bibitem record has three parts: the first part in
% curly braces is how the citation will appear in the text; the second
% is a shorthand tag that you use to refer to the paper by name; and
% the third is the contents of the bibliography entry.  Leave the
% first and third parts alone, but you can edit the second part (in
% square braces) to be whatever abbreviation you want to use to refer
% to that paper - here I've changed the default ones that come from
% ADS to be something I can remember more easily (usually just an
% author and year).  These are the tags that are used above in the
% \citet and \citep commands.

\bibitem[Bary et al.(2002)]{Bary2002}
Bary, J.~S., Weintraub, D.~A., \& Kastner, J.~H.\ 2002,  \apjl, 576,
L73

\bibitem[Kwitter et al.(2013)]{Kwitter2013} Kwitter, K.~B., Lehman,
E.~M.~M., Balick, B., \& Henry, R.~B.~C.\ 2013, \apj, 768, 97

\bibitem[Willman
\& Strader(2012)]{Willman2012} Willman, B., \& Strader, J.\ 2012, \aj, 144, 76


\end{thebibliography}

\end{document}
