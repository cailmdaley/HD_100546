\documentclass[onecolumn]{aastex6}

\newcommand{\vdag}{(v)^\dagger}
\newcommand{\myemaila}{c.walsh1@leeds.ac.uk}
\newcommand{\myemailb}{cwalsh@strw.leidenuniv.nl}

%% Put in packages here
\usepackage{amsmath}
\usepackage{amssymb}
\usepackage{url}
\usepackage{gensymb}
\usepackage{graphics}
\usepackage{subfigure}
\usepackage{wasysym}
\usepackage{textcomp}
\usepackage{xcolor}
\usepackage{multirow}
\usepackage[version=3]{mhchem}
%\usepackage{multicol}
%% End of packages


\newcommand{\kms}{km~s$^{-1}$}
\newcommand{\perbeam}{beam$^{-1}$}

\shorttitle{Kinematics of the HD 100546 disk}
\shortauthors{Catherine Walsh et al.}

%% Begin document
\begin{document}

\title{CONFIRMATION OF A WARPED INNER DISK AROUND HD~100546}
\author{Catherine Walsh\altaffilmark{1,2} \& Cail Daley\altaffilmark{3} \& Attila? \& Stefano?}

% Who else? Attila and Stefano. And Paola?

\altaffiltext{1}{School of Physics and Astronomy, University of Leeds, Leeds, LS2 9JT, UK}
\altaffiltext{2}{Leiden Observatory, Leiden University, P.~O.~Box 9531, 2300~RA Leiden, The Netherlands}
\altaffiltext{3}{Astronomy Department, Wesleyan University, 96 Foss Hill Drive, Middletown, CT 06459, USA}

\email{\myemaila; \myemailb}

\begin{abstract}   
We present spatially resolved Atacama Large Millimeter/Submillimeter Array
(ALMA) images of CO~$\mathrm{J}=3-2$ emission from the protoplanetary disk
around Herbig Ae star HD~100546. We expand upon previous analysis of this data
and model the spatially resolved kinematic structure (i.e. the first moment map)
of the gas disk, fully exploring our parameter space. We uncover significant
deviations from purely Keplerian motion, and suggest the presence of a severely
warped and twisted inner disk extending to roughly 100~au. The inner disk is
nearly edge-on to the line of sight and exhibits a position angle almost
orthogonal to that of the outer disk, which previous work has shown to have a
position angle of $146\degree$ and an inclination of $44\degree$. Our finding
are contrasted by VLT/SPHERE observations which show no evidence of a warp
beyond 10~au. Fast radial inflows may also be capable of producing the observed
deviations from a Keplerian velocity structure. HD~100546 joins several other protoplanetary disks that have been shown to exhibit non-Keplerian kinematics.
\end{abstract}

\section{INTRODUCTION}
\label{introduction}

Observations of protoplanetary disks around nearby young stars offer unique
insight into the initial conditions that govern planetary system formation.
Resolved dust continuum observations spanning optical to cm wavelengths reveal
the spatial distribution of dust across a range of grain sizes, which in turn
can highlight signposts of ongoing planet formation and/or as yet unseen massive
companions/planets (e.g., cavities, gaps, rings, and spirals; see the recent
reviews by \citealt{espaillat14}, \citealt{andrews15}, and \citealt{grady15}).
Likewise, resolved observations of line emission disclose the spatial
distribution of different species of gas and can provide information about
disk kinematics and structure in addition to information about the gas itself (review paper here? I poked around a little bit, but wasn't able to find a recent review of observations of gas in protoplanetary disks).


Second only to to \ce{H2} in gas-phase molecular abundance, CO is a powerful
diagnostic of various properties including the gas disk mass, surface density,
and temperature. The primary isotopologue, $^{12}$CO, is
optically thick and thus emits from the warm disk atmosphere allowing derivation
of the gas temperature in this region \citep[e.g.,][]{williams11,dutrey14}. The
rarer isotopologues ($^{13}$CO, C$^{18}$O, C$^{17}$O and $^{13}$C$^{18}$O) have
progressively lower opacities and so enable penetration towards and into the
disk midplane (see, e.g., recent theoretical studies by \citealt{bruderer13},
\citealt{miotello16}, and \citealt{yu16}). In observations with sufficiently
high spatial resolution, now routine with ALMA, this allows a direct
determination of the location of the CO snowline with high precision \citep[see,
e.g.,][]{nomura16,schwarz16,zhang17}. However, it has been demonstrated recently
that chemistry, in particular isotope-selective photodissociation
\citep{visser09}, can complicate the extraction of disk gas masses from CO
isotopologue emission \citep{miotello14,miotello16}.

Because emission from $^{12}$CO (and often $^{13}$CO) at (sub)mm wavelengths is
bright, it has historically been used as a tracer of disk kinematics allowing a
dynamical determination of the mass of the central star
\citep[e.g.,][]{simon00}. However, gas motion can deviate from that expected
due to solely Keplerian rotation because of a variety of physical effects that
can be inferred from spatially-resolved observations, including the presence
of spiral density waves, a substantial (and thus measurable) gas pressure
gradient, radial flows mediated by accreting planets across cavities, or a disk
warp, \citep[see,
e.g.,][]{rosenfeld12,tang12,casassus13,rosenfeld14,christiaens14,casassus15}.
Spirals, radial flows, and warps can all signify the presence of planetary
companions; hence, perturbations from Keplerian motion traced in bright and
spectrally- and spatially-resolved CO emission can provide evidence for the
presence of unseen planets.

Here, we present high signal-to-noise and spectrally-resolved ALMA Cycle 0
images of $^{12}$CO $\mathrm{J}=3-2$ emission from the protoplanetary disk
around the nearby Herbig~Ae star HD~100546. The HD~100546 disk has been proposed
to host (at least) two massive companions \citep[see,
e.g.,][]{acke06,quanz13,walsh14}. In \citet{walsh14}, henceforth referred to as
Paper I, we presented the $^{12}$CO ($\mathrm{J}=3-2$) first moment map and dust
continuum emission at 302 and 346~GHz respectively. These data spatially
resolved the CO emission and allowed direct determination of the radial extent
of the molecular disk ($\approx 390$~au, see also \citealt{pineda14}). The
continuum data analysed in Paper I showed that the (sub)mm-sized dust grains had
been sculpted into two rings. \citet{pinilla15} showed that this dust morphology
is consistent with dust trapping by two massive companions: one with mass
$\approx 20 M_\mathrm{J}$ at 10~au, and one with mass $\approx 15 M_\mathrm{J}$
at 70~au. Emission from $^{12}$CO ($\mathrm{J}=3-2$, $6-5$, and $7-6$) from
HD~100546 had been detected previously with APEX \citep{panic10}. The APEX data
also revealed an asymmetry in the red and blue peaks in the double-peaked line
profiles most apparent in the $\mathrm{J}=3-2$ and $6-5$ transitions.
\citet{panic10} hypothesised that the asymmetry may arise due to shadowing of
the outer disk by a warp in the inner disk.

In the following pages, we revisit the HD~100546 ALMA Cycle~0 data and  conduct
a deeper analysis of the spatially and spectrally resolved $^{12}$CO
$\mathrm{J}=3-2$ emission. The focus of this work is the search for evidence of
a warp in the inner regions of the disk, as suggested by the single dish data
presented in \citet{panic10}. Using the same data set as here, \citet{pineda14}
showed that the position-velocity (P-V) diagram across the major axis of the
disk is better described by a disk inclination of $\approx 30\degree$, rather
than the $44\degree$ inclination which best reproduces the aspect ratio of the
disk as seen in continuum emission (Paper I). In Section~\ref{almaimages}, we
outline the imaging presented in the paper, and in Section~\ref{modelkinematics}
we describe the modeling techniques used and present the results.
Sections~\ref{discussion} and \ref{conclusion} discuss the implications and
state the conclusions, respectively.

%Analysis of the continuum data in the visibility domain revealed
%that the dust had been sculpted into two rings: one at 26~au with a
%width of 21~au, and a second at 190~au with a width of 75~au.
%The outer ring is significantly fainter than the inner ring with a
%contrast ratio of $\sim100$.
%Dust evolution modelling presented in \citet{walsh14}
%and \citet{pinilla15} confirmed that this emission morphology is consistent with
%radial drift and dust trapping due to the presence of two companions
%(a 20 M$_\mathrm{J}$ companion at 10~au and a 15~M$_\mathrm{J}$ companion at 70~au);
%however, the contrast ratio in the inner and outer dust rings can only be
%reproduced if the outer companion (which may be still forming)
%is significantly younger than the inner companion
%\citep[$\approx 2.5$~Myr, for full details on the modelling see][]{pinilla15}.

%In Paper I, a shallow analysis only was conducted for the
%$^{12}$CO~$\mathrm{J}=3-2$ emission.
%These data resolved, for the first time, the expansive molecular disk around HD~100546
%citep[confirming an outer radius of 390~au, see also the analysis presented in][]{pineda14}.
%The radius of molecular disk extends to $\approx \times 2$ the
%dust continuum emission, additional evidence for the
%radial drift and depletion of mm-sized dust grains from
%the outer regions of the disk.

\section{ALMA~IMAGING~OF~HD~100546 }
\label{almaimages}

HD~100546 was observed with ALMA on 2012 November 24 with
24 antennas in a compact configuration, with baselines
ranging from 21 to 375~m.
The self-calibrated and phase-corrected
measurement set, produced as described in Paper I,
is used in these analyses.
In this work, we adopt the revised distance to
HD~100546 determined by Gaia \citep[$109\pm4$ pc,][]{gaia16a,gaia16b},
and a stellar mass of $2.4 \mathrm{M}_{\odot}$, \citep{vandenancker98}

In Paper I, the integrated intensity
and first moment maps from the $^{12}$CO $\mathrm{J}=3-2$ rotational
transition at 345.795~GHz ($E_\mathrm{up} = 33.19~\mathrm{K}$
and $A_\mathrm{ul} = 2.497\times 10^{-6}~\mathrm{s}^{-1}$) were presented.
The data cube from which those maps were produced was itself produced
using the CASA task \texttt{clean} with Briggs weighting (robust=0.5)
at a spectral resolution of 0.15~\kms.
The resulting channel maps had an rms noise of 19 mJy~\perbeam~channel$^{-1}$
and a synthesised beam of $0\farcs95\times0\farcs42~(38\degree)$.
The $^{12}$CO was strongly detected with a signal-to-noise ratio
(S/N) of 163 in the channel maps.

Because of the high S/N the imaging is redone here
using uniform weighting which results in a smaller beam (and improved
spatial resolution) at the expense of sensitivity.
The resulting channel maps have an rms noise of 26
mJy~beam$^{-1}$~channel$^{-1}$, a S/N of 106, and
a synthesised beam of $0\farcs92\times0\farcs38~(37\degree)$.
Figure~\ref{figure1} presents the channel maps.
Emission is detected ($\ge 3\sigma$) across 111 channels: the central channel
is centred at the source velocity of 5.7~\kms~as constrained
previously by these data (see Paper I).
The highest velocity emission detected is $\pm 8$~\kms relative
to the source velocity.
Given that the disk inclination (as constrained by the outer disk)
is 44\degree~and that the stellar mass is $2.4 \mathrm{M}_\odot$,
emission is detected down to a radius of 15~au from the central star.
Using the estimate of $\approx 30\degree$ for the inclination of the
inner disk from \citet{pineda14}, reduces this radius to 8~au.

The channel maps in Figure~\ref{figure1} reveal the classic `butterfly' morphology of
spectrally- and spatially-resolved line emission from an inclined
and rotating protoplanetary disk \citep[see, e.g.,][]{semenov08}.
Compared with resolved $^{12}$CO emission from other similarly inclined Herbig~Ae disks
at a similar spatial and spectral resolution
(e.g., HD~163296, \citealt{degregorio13} and\citealt{rosenfeld13}, and
HD~97048, \citealt{walsh16} and \citet{vanderplas17}),
there is no evidence of emission from the back side of disk that is a
signature of CO freezeout in the disk midplane coupled with
emission from a flared surface.
The blue-shifted emission also appears symmetric about the disk major axes
indicating that the emission arises from a relatively `flat' surface,
in contrast with other Group I Herbig Ae/Be disks
\citep[e.g., HD~97048,][]{walsh16,vanderplas17}.
However, the emission is not wholly symmetric about the disk minor axis,
with the red-shifted emission from the north-west quadrant appearing both
fainter, and with a positional offset, relative to blue-shifted
emission at the same velocity.
In Figure~\ref{figure2} the channel maps from $\pm 0.45$ to $\pm 1.5$~\kms are
shown, now rotated clockwise to align the disk major axis in the vertical direction, and
mirrored across the disk minor axis.
Exhibiting the data in this velocity range and in this manner highlights
the described asymmetry in brightness across the disk minor axis, the flatness of the emission, and
the positional offset of the red-shifted north-west lobe relative to
its blue-shifted counterpart.

% \begin{figure*}[!htb]
% \includegraphics[width=\textwidth]{./HD100546_CO_3-2_channel_maps.eps}
% \caption{Channel maps of the CO $\mathrm{J}=3-2$ line emission imaged at a
% spectral resolution of 0.15~\kms. The dashed lines
% represent the disk major and minor axes determined from analysis of the
% continuum \citep{walsh14}.}
% \label{figure1}
% \end{figure*}
%
% \begin{figure*}[ht]
% \includegraphics[width=\textwidth]{./HD100546_CO_3-2_mirror_maps.eps}
% \caption{Channel maps of the CO $\mathrm{J}=3-2$ line emission rotated
% to align the disk major axis in the vertical direction, and mirrored across the disk minor axis.}
% \label{figure2}
% \end{figure*}

Figure~\ref{figure3} present the moment maps
(zeroth, first, second, and eighth).
The zeroth moment map (integrated intensity) was produced using
a $3\sigma$ rms noise clip.
The line profile was extracted from a polygon delineated by the
$3\sigma$ spatial extent of the emission in the channel maps.
The first (intensity-weighted velocity), second (intensity-weighted velocity dispersion),
and eighth (peak flux density) moment maps were produced using a more
conservative clip of $6\sigma$.

The integrated intensity appears relatively symmetric about the
disk minor axis; however, the $^{12}$CO integrated emission extends
further to the south-west than it does to the north-east.
This asymmetry is also evident in the eighth-moment map with the
north-east side of the disk appearing brighter than the south-west side.
Both maps hint at emission from a flared disk which would lead to an
asymmetry in emission across the disk major axis (i.e., the axis of inclination).
The first and second moment maps also hint at asymmetric emission, in particular,
the emission at the source velocity through the inner disk is twisted relative to the
disk minor axis determined from the continuum emission.
The velocity dispersion in the inner disk is also not wholly symmetric across
the disk minor axis.
Both images suggest the presence of a warp in the inner disk.

% \begin{figure*}[ht]
% \centering
% \subfigure{\includegraphics[width=0.45\textwidth]{./HD100546_CO_3-2_line_mom0.eps}}
% \subfigure{\includegraphics[width=0.45\textwidth]{./HD100546_CO_3-2_line_mom1.eps}}
% \subfigure{\includegraphics[width=0.45\textwidth]{./HD100546_CO_3-2_line_mom2.eps}}
% \subfigure{\includegraphics[width=0.45\textwidth]{./HD100546_CO_3-2_line_mom8.eps}}
% \caption{Moment maps for CO $\mathrm{J}=3-2$ line emission from HD 100546.  Clockwise from  top
% left: zeroth moment map (integrated intensity, Jy~\perbeam), first moment map (intensity-weighted
% velocity, \kms), eighth moment map (peak intensity, Jy~\perbeam), and second moment map
% (intensity-weighted velocity dispersion, \kms).
% The dashed contour in the second and eighth moment maps corresponds to the $3\sigma$ contour
% of the integrated intensity.}
% \label{figure3}
% \end{figure*}

\section{MODELLING THE KINEMATICS}
\label{modelkinematics}

Modeling of the kinematics as traced by the $^{12}$CO emission is conducted
using analytical models which describe the line-of-sight projected velocity,
i.e. the first moment map.
The model moment maps are convolved with the synthesised beam of the observations.
The residuals ($\mathrm{data} - \mathrm{model}$) are in units of a single spectral
resolution element ($\delta v = 0.21$~\kms).
This discretisation is necessary because features smaller than the native spectral
resolution of the data cannot be fit.
We explore several metrics of `best fit':
(i) the total number of pixels for which the analytical and smoothed
projected velocity reproduces the data within one spectral resolution element,
(ii) the sum of the square of the residuals divided by the total number of pixels, and
(iii) the magnitude of the peak residual.
The total number of unmasked pixels in the observed first moment map is 2562.

Given the relatively small number of parameters for each model considered,
the modelling approach is grid based,
i.e. all possible grid combinations are explored.

\subsection{A flat emitting surface}

The simplest prescription for describing the first moment map of spectrally-resolved
line emission from a disk
is axisymmetric emission arising from a geometrically flat surface inclined to the line of sight.
Assuming that the position angle of the disk is aligned with the
y axis, the projected velocity on the sky relative to the observer is described by
\begin{equation}
v(x',y') = \sqrt{\frac{G M_\star}{r}} \sin i \sin \theta,
\label{velocity}
\end{equation}
where $G$ is the gravitational constant, $M_\star$ is the mass of the central
star, $r=\sqrt{x^2+y^2}$ is the radius, $i$ is the inclination, and
$\theta = \arctan(y/x)$ \citep[e.g.,][]{rosenfeld13}.
In this projection and for this particular orientation,
$x = x'/\cos i$, $y = y'$, and $z = 0$.
Model first moment maps for a flat disk with the same P.A as
HD~100546 at three different inclinations, 30\degree, 45\degree, and
60\degree, are shown in Figure~\ref{figurea1} in the Appendix.

The wide range of disk inclinations ($[20\degree,60\degree]$) and disk
position angles ($[120\degree,170\degree]$) explored are motivated by previous analyses
of the continuum data which suggested a P.A.~of $146\degree \pm 4\degree$
and an inclination of $44\degree \pm 3\degree$ \citep[see Paper I and][]{pineda14}.
Using the CO data, \citet{pineda14} suggest that the inner disk may be better described
with an inclination of $\approx 30\degree$ so we extend our explored range
accordingly to ensure good coverage over the parameter space.
First, a coarse grid with resolution 5\degree~is run over the full parameter space,
followed by a zoomed in region with resolution 1\degree.

The top-left panel of Figure~\ref{figure4} presents a 3D plot showing the
total number of pixels which fit the data velocity field within one spectral resolution
element, $\delta v$, as a function of disk inclination and position angle.
The distribution is strongly peaked: the best-fit flat disk model using this
metric has an inclination of 36\degree~ and a P.A.~of 145~\degree~with 62.1\% of model pixels
lying within one spectral resolution element of the data.
These data are also listed in Table~\ref{table1}.
The P.A.~is in excellent agreement with that derived from the continuum observations.
The inclination, on the other hand, is lower and closer
to the suggested inclination from \citet{pineda14}.

The left hand plots of Figure~\ref{figure5} show the distribution of residuals summed
over the entire disk (top panel) and the residual first moment map
(bottom panel).
The histogram of residuals shows small dispersion about 0 with 96.0\% of pixels
matching the data within $\pm0.315$ \kms).
The residual map shows that a flat disk well reproduces the large-scale
velocity field: the largest deviations from this model occur in the innermost
disk where the model velocity field over-predicts the magnitude (by up to $7\delta v$) of the
projected line-of-sight velocity along the minor axis of the disk.
This leads to negative residuals in the north-east
and positive residuals in the south-west.
The morphology of the residuals suggests that the inner disk has an
additional inclination along the minor axis of the outer disk, i.e.,
close to orthogonal to that of the outer disk.

% \begin{figure*}[]
% \subfigure{\includegraphics[width=0.33\textwidth]{./flat_stats.eps}}
% \subfigure{\includegraphics[width=0.33\textwidth]{./flared_lower_stats.eps}}
% \subfigure{\includegraphics[width=0.33\textwidth]{./flared_upper_stats.eps}}
% \caption{Distribution of model best-fit values using metric (i) as a function of
% inclination and position angle for the best-fit flat disk, flared disk (lower cone),
% and flared disk (upper cone), respectively.
% The best-fit opening angles, $\alpha$, of the flared disks (with respect to the
% disk midplane) are 13\degree~and 9\degree~for the lower and upper cones, respectively.
% The percentage scale corresponds to the full range of pixel values (from 0 to 2652).}
% \label{figure4}
% \end{figure*}
%
% \begin{figure*}[]
% \subfigure{\includegraphics[width=0.33\textwidth]{./upper_hist_i36_p145_a00.eps}}
% \subfigure{\includegraphics[width=0.33\textwidth]{./lower_hist_i38_p142_a13.eps}}
% \subfigure{\includegraphics[width=0.33\textwidth]{./upper_hist_i36_p145_a09.eps}}
% \subfigure{\includegraphics[width=0.33\textwidth]{./upper_res_i36_p145_a00.eps}}
% \subfigure{\includegraphics[width=0.33\textwidth]{./lower_res_i38_p142_a13.eps}}
% \subfigure{\includegraphics[width=0.33\textwidth]{./upper_res_i36_p145_a09.eps}}
% \caption{Residual histograms (top) and maps (bottom)
% using metric (i) as the measure of best fit for a geometrically flat disk (left),
% and the lower cone (middle) and upper cone (right) of a flared disk.
% The histograms are displayed on a log scale to emphasise the largest residuals.}
% \label{figure5}
% \end{figure*}

\subsection{A flared emitting surface}

Although a geometrically flat disk well reproduces much of the
the velocity field, particularly for the outer disk,
we test next whether emission from a flared surface can improve upon the fit.
This is important to check because HD~100546 is classified as a Group I (i.e., flared)
Herbig Ae star \citep{meeus01}, so one might expect the $^{12}$CO emission to arise from a layer
higher up in the disk atmosphere.
Indeed, thermo-chemical modelling of the disk around HD~100546 by \citet{bruderer12}
suggests that the $^{12}$CO line emission arises from a layer $z/r \approx 0.2$.

\citet{rosenfeld13} modelled the emission from the disk around
the Herbig~Ae star HD~163296 by assuming that the emission arises from
an inclined and flared surface with some opening angle, $\phi$, relative
to the $(x,y)$ plane (the disk midplane), i.e., a `double-cone' morphology.
In this way, the front and back sides of the disk with the same projected
line-of-sight velocity are spatially offset \citep[see also][]{degregorio13}.
Here, a similar toy model is used; however, to determine the
line-of-sight velocity, the radius defined in spherical coordinates
($r=\sqrt{x^2+y^2+z^2}$) is adopted rather than that defined in cylindrical coordinates
($\rho=\sqrt{x^2+y^2}$, as was done in \citealt{rosenfeld13}).
For small opening angles the two methods give similar results
(the radii differ by no more than 10\% for $\phi \le 25\degree$).
A flared disk with this emission morphology has two possible orientations with either
the surface of lower cone or upper cone facing the observer
(see e.g., figure 3 in \citealt{rosenfeld13}).
Model first moment maps for a flared disk with the same P.A and inclination as
HD~100546, but with three different opening angles, 20\degree, 45\degree, and
60\degree, are shown in Figure~\ref{figurea2} in the Appendix.

The range of surface opening angles ($[0\degree,20\degree]$) is motivated by
previous thermo-chemcial
modelling of CO emission from HD~100546
which suggest an opening angle $\phi \approx 11\degree$ \citep[][]{bruderer12}.
The symmetry in the channel maps (Figure~\ref{figure3}) also suggests that the emitting layer
lies relatively close to the midplane.
As before, a coarse grid with a resolution of 5\degree~is initially
run over the full parameter space, followed by a zoomed in grid
with a resolution of 1\degree.

Figures~\ref{figure4} and \ref{figure5} present the statistics and residuals
for the best-fit lower cone and upper cone of a flared disk.
Using metric (i), the best-fit upper cone model fits the data marginally better
(reproducing 65.0\% of the velocity field) than both the flat disk and the best-fit lower
cone model (62.1\% and 62.6\%, respectively, see Table~\ref{table1}).
The best-fit inclination, P.A., and opening angle are 36\degree,
145\degree, and 9\degree, respectively (see Table~\ref{table1}).
The opening angle of the $^{12}$CO-emitting surface agrees well with that
suggested from thermo-chemical models of HD~100546 \citep{bruderer12}.

The best-fit lower cone model has an inclination of 38\degree,
a P.A.~of 142\degree, and an opening angle of 13\degree.
The inclination of this model lies closest to that derived from the
continuum observations ($44\degree \pm 3\degree$).
Despite resulting in a marginally worse fit to the data than
the upper cone model (see Table~\ref{table1}), a `by-eye'
examination of the residual map (bottom left panel of
Figure~\ref{figure5}) shows that this
morphology best reproduces the velocity field in all quadrants
of the outer disk excepting the north-west quadrant for
which the magnitude of velocity field is over-estimated.
Comparing this residual map to both the
channel map (Figure~\ref{figure1}) and the
eighth moment map (bottom right panel of Figure~\ref{figure3})
shows that emission from this quadrant appears less bright
and exhibits a positional offset relative to
that mirrored across the minor axis of the disk.
However, that the upper cone model fits the data best using this metric
is in agreement with recent VLT/SHERE images that confirm that the
the far side of the flared disk surface lies towards the north \citep{garufi16}.

For all three models, the global best-fit using metric (ii), i.e.,
sum of the squares of the residuals normalised by the total number of pixels,
is a flat disk with an inclination of 37\degree~and a P.A.~of 142\degree.
The best-ft model selected by the smallest peak residual,
i.e., metric (iii), is also shared by all three models and
is a flat disk with an inclination of 39\degree~(again in good agreement
with the other two metrics); however, the disk P.A which gives the
smallest peak residual is 126\degree.
The residual histograms and maps for both of these models are
shown in Figure~\ref{figure6}.
That the inner disk velocity structure is better fit with a shallower
P.A.~than the outer disk, highlights the presence of a twisted warp:
this is investigated in the subsequent section.

\begin{deluxetable*}{lcccccccc}
\tablecaption{Best-fit parameters for the flat and flared kinematic models. \label{table1}}
\tablewidth{0pt}
\tablehead{\colhead{Model} & \colhead{Metric}  & \colhead {Inclination} & \colhead{P.A.} & \colhead{Opening}
& \colhead{Pixel} & \colhead{Percentage}  & \colhead{Sum of} & \colhead{Peak} \\
  & \colhead{of best-fit} &  &  & \colhead{angle}
& \colhead{number} &    & \colhead{residual squares} & \colhead{residual}}
\startdata
\hline
Flat                & (i)   & 36\degree & 145\degree & 0\degree  & 1590 & 62.1\%  & 0.737 & 7.23 \\
                    & (ii)  & 37\degree & 142\degree & 0\degree  & 1550 & 60.5\%  & 0.665 & 6.71 \\
                    & (iii) & 39\degree & 126\degree & 0\degree  &  224 &  8.8\%  & 3.610 & 3.63 \\
Flared (lower cone) & (i)   & 38\degree & 142\degree & 13\degree & 1605 & 62.6\%  & 0.548 & 6.64 \\
Flared (upper cone) & (i)   & 36\degree & 145\degree & 9\degree  & 1665 & 65.0\%  & 0.764 & 7.28 \\
\enddata
\end{deluxetable*}

% \begin{figure*}[]
% \centering
% \subfigure{\includegraphics[width=0.33\textwidth]{./upper_hist_i37_p142_a00.eps}}
% \subfigure{\includegraphics[width=0.33\textwidth]{./upper_hist_i39_p126_a00.eps}}
% \subfigure{\includegraphics[width=0.33\textwidth]{./upper_res_i37_p142_a00.eps}}
% \subfigure{\includegraphics[width=0.33\textwidth]{./upper_res_i39_p126_a00.eps}}
% \caption{Residual histograms (top) and maps (bottom) for a geometrically flat, rotating disk,
% using metrics (ii) and (iii) as the measure of best fit.
% The histograms are displayed on a log scale to emphasise the largest residuals.}
% \label{figure6}
% \end{figure*}

\subsection{A warped disk}

The residual maps displayed in Figures~\ref{figure5} and \ref{figure6}
reveal two features: (i) a rotating disk within $\approx$1\farcs0 of the source
position with an inclination angle approximately orthogonal to that of the outer disk, and
(ii) a shallower position angle on small scales ($\lesssim$1\farcs0) than on larger scales.
Both results point towards a twisted warp in the inner disk
\citep[see, e.g,][and references therein]{juhasz17}.

Because the residuals are of the order of the size of the synthesised beam,
a simple toy prescription for the warp is used.
The inner disk is modelled as a planar disk within a fixed radius which
possesses its own inclination and P.A, i.e., the inner disk is
misaligned relative to the outer disk.
This is similar to the approach used by \citet{rosenfeld14} to
model the kinematics of HD~142527.
Figure~\ref{figurea3} presents model first moment maps for a warped
disk for a range of inclinations and position angles, and for a fixed
transition radius of 100~au and outer disk parameters
appropriate for the HD~100546 disk.

The outer disk velocity structure is fixed to that of the best-fit upper cone model.
As mentioned in the previous section, recent VLT/SPHERE images of scattered light
from HD~100546 suggest that the far side of the (flared) disk lies towards the
north-east \citep{garufi16}.
This results in three additional fitting parameters only:
the inner disk inclination ([40\degree, 90\degree]), the inner disk P.A.~([40\degree, 100\degree]), and
a transition radius marking the boundary between the inner
and outer disks ([40, 120]~au).
A coarse grid with a resolution of 10\degree~and 10~au is first run to
identify the parameter space containing the global best-fit, followed by
a finer grid over this zoomed-in region (with a resolution of 2\degree~and 2~au).

Figures~\ref{figure7} and \ref{figure8} present the statistics and residuals for
the best-fit warped disk, respectively.
Metric (i) favours a model with an inner disk that is almost `edge-on'
($i=80\degree$) to the line of sight, almost orthogonal to the outer disk
major axis (P.A.$= 60\degree$), and with a transition radius of 90~au
(see Figure~\ref{figure7} and Table~\ref{table2}).
These values are consistent with the morphology of the residuals of both
the flat and flared models (see Figures~\ref{figure5} and \ref{figure6}).
The magnitude of the peak residual of this model is significantly smaller than
the previous models selected using metric (i), $4 \delta v$ versus $7 \delta v$.
Metrics (ii) and (iii) select the same model (see Table~\ref{table2})
with parameters similar to those using metric (i); an inclination of
84\degree, a P.A. of 64\degree, and a transition radius of 100~au.
Comparing the residual histograms and maps for these two models (shown in
Figure~\ref{figure8}), highlights how a small change in inclination and/or
position angle can significantly reduce the magnitude of the peak residual.
This latter model results in a peak residual of only $2.4 \delta v$ and has
the smallest dispersion of residuals: 98\% of pixels match the data within
$\pm 0.315$~\kms~and 100\% of pixels match within $\pm 0.525$ \kms.

% \begin{figure}[]
% \centering
% \subfigure{\includegraphics[width=0.33\textwidth]{./warp_stats.eps}}
% \caption{Distribution of model best-fit values using metric (i) as a function of
% inclination and position angle for the best-fit warped disk.
% The best-fit transition radius using this metric is 90~au.
% In this plot, the percentage scale corresponds to the z-axis range.}
% \label{figure7}
% \end{figure}
%
% \begin{figure*}[]
% \centering
% \subfigure{\includegraphics[width=0.33\textwidth]{./warp_hist_r090_i080_p060.eps}}
% \subfigure{\includegraphics[width=0.33\textwidth]{./warp_hist_r100_i084_p064.eps}}
% \subfigure{\includegraphics[width=0.33\textwidth]{./warp_res_r090_i080_p060.eps}}
% \subfigure{\includegraphics[width=0.33\textwidth]{./warp_res_r100_i084_p064.eps}}
% \caption{Residual histograms (top) and maps (bottom) for a protoplanetary disk
% with a warped inner disk using metrics (i) and (ii) as the metric of best fit.
% The histograms are displayed on a log scale to emphasise the largest residuals.
% Note that metrics (ii) and (iii) select the same warped inner disk parameters
% (see Table 2).}
% \label{figure8}
% \end{figure*}

\begin{deluxetable*}{lcccccccc}
\tablecaption{Best-fit parameters for the warped kinematic models. \label{table2}}
\tablewidth{0pt}
\tablehead{\colhead{Model} & \colhead{Metric}  & \colhead {Inclination} & \colhead{P.A.} & \colhead{Transition}
& \colhead{Pixel} & \colhead{Percentage}  & \colhead{Sum of} & \colhead{Peak} \\
  & \colhead{of best-fit} &  &  & \colhead{radius (au)}
& \colhead{number} &    & \colhead{residual squares} & \colhead{residual}}
\startdata
\hline
Warped              & (i)            &  80\degree &  60\degree &  90 & 1722  & 67.2\%  & 0.387 & 4.064 \\
                    & (ii) \& (iii)  &  84\degree &  64\degree & 100 & 1710  & 66.4\%  & 0.350 & 2.441 \\
\enddata
\end{deluxetable*}

\section{DISCUSSION}
\label{discussion}

The analysis of the kinematics of the HD~100546 protoplanetary disk (traced in
$^{12}$CO $\mathrm{J}=3-2$ emission with ALMA) presented here has revealed the
presence of a severely warped (misaligned) inner disk within 100~au of the
central star. The inner disk exhibits a position angle almost orthogonal to that
of the outer disk, and is inclined almost edge-on to the line-of-sight.
Figure~\ref{figure9} shows an idealised model of the twisted and warped
HD~100546 protoplanetary disk. Higher spatial resolution data are needed to
confirm if the proposed warp lies within a smaller radial region than suggested
by these data; scattered light images taken with VLT/SPHERE with a spatial
resolution of 0\farcs02 reveal no evidence of a severely misaligned dust disk
beyond 10~au \citep{garufi16}. Hence, if the warp is triggered by the presence
an inner companion in HD~100546, then the warp also lies within 10~au of the
central star (why this conditional? The VLT observations seem to indicate that
the warp lies within 10 au regardless of the underlying mechanism, as long as
the gas and dust are cospatial). Alternatively, the dust disk and gas disk are
tracing different physical processes, which is discussed further below.

The presence of a disk warp in HD~100546 was originally proposed by
\citep{panic10} and based on asymmetries seen in the red and blue peaks of
single-dish spectra observed with APEX. The ALMA data presented here show no
such asymmetries in the peaks of the line profile (see Figure~\ref{figure10});
however, the mirrored line profile does highlight that the red and blue lobes of
emission have different shapes when integrated over the disk. This is consistent
with the morphology seen in the channel maps and described in
Section~\ref{almaimages}.

%\begin{figure}[]
%\includegraphics[width=0.5\textwidth]{./}
%\caption{Cartoon of the warped HD~100546 protoplanetary disk.}
%\label{figure9}
%\end{figure}

% \begin{figure}[]
% \centering
% \includegraphics[width=0.5\textwidth]{./CO_346GHz_line_profile.eps}
% \caption{CO $\mathrm{J}=3-2$ line profile extracted from within the
% $3\sigma$ contour of the integrated intensity (see Figure~\ref{figure3}).}
% \label{figure10}
% \end{figure}

The presence of a warp in the inner disk could shadow the outer disk and lead to
an asymmetric temperature structure, as one side of the side is more illuminated
by the central star than the opposite side (see figure 3 in \citealt{panic10}).
Because the $^{12}$CO emission is optically thick, it nicely traces the gas
temperature, and thus any perturbations, in the disk atmosphere. Given that the
warp is likely to be triggered by the presence of a massive companion in orbital
motion about the central star, the warp is expected to orbit with a similar
period to that of the companion (as proposed for the case of TW Hya by
\citealt{debes17}) and should generate a similar periodicity in the shape of the
$^{12}$CO line profiles.

The APEX $^{12}$CO $\mathrm{J}=3-2$ line profile from \citet{panic10} shows a
stronger blue peak. The dates of the observations taken with APEX and ALMA are
November 2008 \citep[][]{panic10} and November 2012 \citep[][]{walsh14},
respectively. There also exist multiple APEX observations of the $^{12}$CO
$\mathrm{J}=6-5$ transition. This line profile presented in \citep{panic10}
shows an even higher-contrast asymmetry in the blue and red peaks than in the
$\mathrm{J}=3-2$ line. More recent observations presented in \citet{kama16}, and
taken between April and July 2014, show a similar asymmetry; however, the ratio
between the peaks is within the flux density uncertainties of the observations
(M.~Kama, priv.~commun.), so it is currently difficult to draw concrete
conclusions on any periodicity in the shapes of the line profiles. A massive
($\approx 10 M_\mathrm{J}$) planetary companion has been proposed to reside at
$\approx 10$~au in HD~100546 and is thought to be responsible for clearing the
inner cavity in both the dust and gas
\citep{mulders13,panic14,walsh14,pinilla15,wright15,garufi16}, as well as
dynamical perturbations in [OI], OH, and CO line emission
\citep{acke06,vanderplas09,brittain14}. A planet at 10~au would have a long
period, $\approx 32$~years; as such, if this companion is indeed responsible for
the warp then observations over much longer timescales are needed.

Similar analyses have revealed warps in disks around other stars.
\citet{rosenfeld12} showed that the kinematic structure of TW Hya
could be reproduced using a parametric model of a disk warp
with a moderate inclination ($\approx~8\degree$) at a radius of 5~au.
\citet{facchini14} demonstrated the observed warp amplitude
could be induced by a misaligned close-in companion as massive as
$\approx 14M_\mathrm{J}$ orbiting within 4~au of the central star.
New scattered light data from HST/STIS, coupled with
archival data over a 17-year baseline, reveal an orbiting
azimuthal brightness asymmetry, with a
period of $\approx 16$~yr \citep{debes17}.
The authors argue that this is consistent with partial
shadowing of the outer disk by a misaligned inner
disk interior to 1~au and that is precessing due to
the presence of a $\approx M_\mathrm{J}$-mass planetary companion.

HD~142527 is another protoplanetary disk in which  non-Keplerian motion, traced
by \ce{HCO+} ($\mathrm{J}=4-3$) and $^{12}$CO ($\mathrm{J}=6-5$) emission, has
been suggested \citep{casassus13,casassus15}. HD~142527 is a transition disk
which possesses the largest known dust cavity in sub(mm) emission
\citep[$\approx 140$~au][]{casassus13,fukagawa13}. The observed gas motions were
postulated to arise from fast (near free-fall) radial flows across the cavity
and accretion onto the central star via a severely misaligned inner disk
\citep{casassus13,casassus15,marino15}. \citet{marino15} propose that that inner
and outer disk are misaligned by 70\degree. However, \citet{rosenfeld14}
conducted a theoretical study of fast radial flows in transition disks and
showed that the derived velocity profile can also mimic that characteristic of
an inner disk warp. Better data at higher spatial and spectral resolution are
required to determine whether this is an alternative explanation for the inner
disk kinematic structure of HD~100546. Indeed, `hot-off-the-press' ALMA
observations of \ce{HCO+} ($\mathrm{J}=3-2$) line emission from AA~Tau at an
angular resolution of 0\farcs2, also show evidence of an inner warp (or fast
radial inflow) in the inner regions revealed by a twist in the first moment map
\citep{loomis17}.

\section{CONCLUSION}
\label{conclusion}

The data and analyses shown here demonstrate clearly that the kinematic
structure of the disk around HD~100546 cannot be described by a purely Keplerian
velocity profile with a universal inclination and position angle. Given the
evidence for the presence of (at least) one massive planetary companion orbiting
within 10~au of the central star, the presence of a disk warp, albeit with
extreme parameters, is currently the most plausible explanation in the absence
of any evidence to the contrary. However, the known morphology of the system
constrains the radius of the warp to within 10~au if the gas and dust are
cospatial; as such, higher spatial resolution data are vital for corroborating
the warp hypothesis. To determine whether the alternative explanation of fast radial flows could be responsible for the observed kinematic structure of HD~100546, higher spatial {\em and} spectral resolution data are required to resolve gas emission across the inner dust cavity ($\lesssim 25$~au) and explore any spatial association with the proposed planetary candidate.


\begin{thebibliography}{}
\bibitem[Acke \& van den Ancker(2006)]{acke06} Acke, B. \& van de Ancker, M.~E.~2006, \aap, 449, 267
\bibitem[van den Ancker et al.(1998)]{vandenancker98} van~den~Ancker, M.~E., de~Winter, D., Tijn A Djie, H.~R.~E., \aap, 330, 145
\bibitem[Andrews(2015)]{andrews15} Andrews, S.~M. 2015, \pasp, 127, 961
\bibitem[Brittain et al.(2009)]{brittain09} Brittain, S.~D., Najita, J.~R., \& Carr, J.~S. 2009, \apj, 702, 85
\bibitem[Brittain et al.(2014)]{brittain14} Brittain, S.~D., Carr, J.~S., Najita, J.~R., Quanz, S.~P., \& Meyer, M.~R. \apj, 791, 136
\bibitem[Bruderer et al.(2012)]{bruderer12} Bruderer, S., van~Dishoeck, E.~F., Doty, S.~D., \& Herczeg, G. 2012, \aap, 541, 91
\bibitem[Bruderer(2013)]{bruderer13} Bruderer, S. 2013, \aap, 559, A46
\bibitem[Casassus et al.(2013)]{casassus13} Casassus, S., van der Plas, G., P\'{e}rez, S.~M., et al.~2013, Nature, 493, 191
\bibitem[Casassus et al.(2015)]{casassus15} Casassus, S., Marino, S., P\'{e}rex, S., et al.~2015, \apj, 811, 92
\bibitem[Christiaens et al.(2014)]{christiaens14} Christiaens, V., Casassus, S., Perez, S., van der Plas, G., \& M\'{e}nard, F. 2014, \apjl, 785,L12
\bibitem[Debes et al.(2017)]{debes17} Debes, J.~H., Poteet, C.~A., Jang-Condell, H., et al.~2017, \apj, 835, 205
\bibitem[Dutrey et al.(2014)]{dutrey14} Dutrey, A., Semenov, D., Chapillon, E., et al.~2014, in Protostars and Planets VI, ed. H. Beuther et al.~(Tucson, AZ: Univ. Arizona Press), 317
\bibitem[Espaillat et al.(2014)]{espaillat14} Espaillat, C., Muzurolle, J., Najita, J., et al.~2014, in Protostars and Planets VI, ed. H. Beuther et al.~(Tucson, AZ: Univ. Arizona Press), 497
\bibitem[Facchini et al.(2014)]{facchini14} Facchini, S., Ricci, L., \& Lodato, G. 2014, \mnras, 442, 3700
\bibitem[Fukagawa et al.(2013)]{fukagawa13} Fukagawa, M., Tsukagoshi, T., Momose, M., et al.~2013, \pasj, 65, L14
\bibitem[Gaia Collaboration(2016a)]{gaia16a} Gaia Collaboration et al.~2016a, \aap, 595, A1
\bibitem[Gaia Collaboration(2016b)]{gaia16b} Gaia Collaboration et al.~2016b, \aap, 595, A2
\bibitem[Garufi et al.(2016)]{garufi16} Garufi, A., Quanz, S.~P., Schmid, H.~M., et al. 2016, \aap, 588, A8
\bibitem[Grady et al.(2015)]{grady15} Grady, C., Fukagawa, M., Maruta, Y., et al.~2015, \apss, 355, 253
\bibitem[de Gregorio Monsalvo et al.(2013)]{degregorio13} de Gregorio Monsalvo, I., M\'{e}nard, F., Dent, B., et al.~2013, \aap, 557, 133
\bibitem[Juh\'{a}sz \& Facchini(2017)]{juhasz17} Juh\'{a}sz, A. \& Facchini, S. 2017, \mnras, in press
\bibitem[Kama et al.(2016)]{kama16} Kama, M., Bruderer, S., van~Dishoeck, E.~F., et al.2016, \aap, 592, A83
\bibitem[Loomis et al.(2017)]{loomis17} Loomis, R.~A., \"{O}berg, K.~I., Andrews, S.~M., \& MacGregor, M.~A. 2017, \apj, in press (arXiv : 1704.02006)
\bibitem[Marino et al.(2015)]{marino15} Marino, S., P\'{e}rez, S., \& Casassus, S. 2015, \apjl,  798, L44
\bibitem[Meeus et al.(2001)]{meeus01} Meeus, G., Waters, L.~B.~F.~M., Bouwman, J., et al.~2001, \aap, 365, 476
\bibitem[Miotello et al.(2014)]{miotello14} Miotello, A., Bruderer, S., \& van~Dishoeck, E.~D.~2014, \aap, 572, A96
\bibitem[Miotello et al.(2016)]{miotello16} Miotello, A., van~Dishoeck, E.~D., Kama, M., \& Bruderer, S.~2016, \aap, 594, A85
\bibitem[Mulders et al.(2013)]{mulders13} Mulders, G.~D., Paardekooper, S.-J., Pani\'{c}, O., et al. 2013, 557, 68
\bibitem[Nomura et al.(2016)]{nomura16} Nomura, H., Tsukagoshi, T., Kawabe, R., et al.~2016, \apjl, 819, L7
\bibitem[Pani\'{c} et al.(2010)]{panic10} Pani\'{c}, O., van~Dishoeck, E.~F., Hogerheijde, M., et al.~2010, \aap, 519, 110
\bibitem[Pani\'{c} et al.(2014)]{panic14} Pani\'{c}, O., Ratzka, Th., Mulders, G.~D., et al. 2014, 562, A101
\bibitem[Perez et al.(2015)]{perez15} Perez, S., Dunhill, A., Casassus, S., et al.~2015, \apj, 811, L5
\bibitem[Pineda et al.(2014)]{pineda14} Pineda, J., Quanz, S.~P., Meru, F., et al.~2014, \apjl, 788, L34
\bibitem[Pinilla et al.(2015)]{pinilla15} Pinilla, P., Birnstiel, T., \& Walsh, C.~2015, \aap, 580, A105
\bibitem[van der Plas et al.(2009)]{vanderplas09} van~der~Plas, G., van de Ancker, M. E., Acke, B., et al. 2009, \aap, 500, 1137
\bibitem[van der Plas et al.(2017)]{vanderplas17}  van~der~Plas, G., Wright, C.~M., M\'{e}nard, F., et al.~2017, \aap, 597, A32
\bibitem[Quanz et al.(2013)]{quanz13} Quanz, S., Amara, A., Meyer, M.~R., Kenworthy, M.~A., Kasper, M., \& Girard, J.~A.~2013, \apjl, 766, L1
\bibitem[Rosenfeld et al.(2012)]{rosenfeld12} Rosenfeld, K.~A., Qi, C., Andrews, S.~M., et al.~2012, \apj, 757, 129
\bibitem[Rosenfeld et al.(2013)]{rosenfeld13} Rosenfeld, K.~A., Andrews, S.~M., Hughes, A.~M., Wilner, D.~J., \& Qi, C.~2012, \apj, 757, 129
\bibitem[Rosenfeld et al.(2014)]{rosenfeld14} Rosenfeld, K.~A., Chiang, E., \& Andrews, S.~M. 2014, \apj, 782, 62
\bibitem[Schwarz et al.(2016)]{schwarz16} Schwarz, K.~R., Bergin, E.~A., Cleeves, L.~I., et al.~\apj, 823, 91
\bibitem[Simon et al.(2000)]{simon00} Simon, M., Dutrey, A., \& Guilloteau, S. 2000, \apj, 545, 1034
\bibitem[Semenov et al.(2008)]{semenov08} Semenov, D., Pavlyuchenkov, Ya., Henning, Th., Wolf, S., \& Launhardt, R. 2008, \apjl, 673, L195
\bibitem[Tang et al.(2012)]{tang12} Tang, Y.-W., Guilloteau, S., Pi\'{e}tu, V., et al. 2012, \aap, 547, A84
\bibitem[Visser et al.(2009)]{visser09} Visser, R., van~Dishoeck, E.~F., \& Black, J.~H. 2009, \aap, 503, 323
\bibitem[Williams \& Cieza(2011)]{williams11} Williams, J. P. \& Cieza, L.~A. 2011, \araa, 49, 67
\bibitem[Walsh et al.(2014)]{walsh14} Walsh, C., Juh\'{a}sz, A., Pinilla, P., et al.~2014, \apjl, 791, L6
\bibitem[Walsh et al.(2016)]{walsh16} Walsh. C., Juh\'{a}sz, A., Meeus, G., et al.~2016, \apj, 831, 200
\bibitem[Wright et a.(2015)]{wright15} Wright, C.~M., Maddison, S.~T., Wilner, D.~J., et al. 2015, \mnras, 453, 414
\bibitem[Yu et al.(2016)]{yu16} Yu, M., Willacy, K., Dodson-Robinson, S.~E., Turner, N., \& Evans II, N.~J. 2016, \apj, 822, 53
\bibitem[Zhang et al.(2017)]{zhang17} Zhang, K., et al.~2017, Nature Astronomy, in press
\end{thebibliography}

\appendix

\section{Model first moment maps}

Figure~\ref{figurea1} shows the first moment maps for a geometrically
flat protoplanetary disk for a range of disk inclinations.
Figure~\ref{figurea2} shows model first moment maps for a flared
disk with a fixed inclination for a range opening angles, $\alpha$, of the emitting
surface.
Figure~\ref{figurea3} shows model first moment maps for a warped inner disk
with a fixed transition radius of 100~au, and a range
of warp inclinations and position angles.
The outer disk parameters are those from the best-fit upper cone model
(i.e., an inclination of 36\degree, a P.A.~of 145\degree, and an opening angle
of 9\degree).

% \begin{figure*}[]
% \subfigure{\includegraphics[width=0.33\textwidth]{./upper_mom1_i30_p145_a00.eps}}
% \subfigure{\includegraphics[width=0.33\textwidth]{./upper_mom1_i45_p145_a00.eps}}
% \subfigure{\includegraphics[width=0.33\textwidth]{./upper_mom1_i60_p145_a00.eps}}
% \caption{Model first moment maps for a geometrically flat disk with the a P.A.~of
% 145\degree~(representative of the HD~100546 disk) at an inclination of 30\degree~(left),
% 45\degree~(middle), and 60\degree~(right).
% The contours are in units of a single spectral resolution element (0.21 \kms).}
% \label{figurea1}
% \end{figure*}
%
% \begin{figure*}[]
% \subfigure{\includegraphics[width=0.33\textwidth]{./lower_mom1_i40_p145_a20.eps}}
% \subfigure{\includegraphics[width=0.33\textwidth]{./lower_mom1_i40_p145_a35.eps}}
% \subfigure{\includegraphics[width=0.33\textwidth]{./lower_mom1_i40_p145_a50.eps}}
% \subfigure{\includegraphics[width=0.33\textwidth]{./upper_mom1_i40_p145_a20.eps}}
% \subfigure{\includegraphics[width=0.33\textwidth]{./upper_mom1_i40_p145_a35.eps}}
% \subfigure{\includegraphics[width=0.33\textwidth]{./upper_mom1_i40_p145_a50.eps}}
% \caption{Model first moment maps for a flared disk (lower and upper cones) with a P.A.~of
% 145\degree~and an inclination of 40\degree~(representative of the HD~100546 disk),
% with an opening angle of 20\degree~(left), 35\degree~(middle), and 50\degree~(right).
% The contours are in units of a single spectral resolution element (0.21 \kms).}
% \label{figurea2}
% \end{figure*}
%
% \begin{figure*}[]
% \subfigure{\includegraphics[width=0.33\textwidth]{./warp_mom1_r100_i025_p055.eps}}
% \subfigure{\includegraphics[width=0.33\textwidth]{./warp_mom1_r100_i050_p055.eps}}
% \subfigure{\includegraphics[width=0.33\textwidth]{./warp_mom1_r100_i075_p055.eps}}
% \subfigure{\includegraphics[width=0.33\textwidth]{./warp_mom1_r100_i080_p025.eps}}
% \subfigure{\includegraphics[width=0.33\textwidth]{./warp_mom1_r100_i080_p055.eps}}
% \subfigure{\includegraphics[width=0.33\textwidth]{./warp_mom1_r100_i080_p085.eps}}
% \caption{Model first moment maps for a warped disk with an outer disk P.A.~of
% 145\degree, an inclination of 36\degree~, and an opening angle of 9\degree.
% The top row of models have an inner warp with a fixed P.A.~of 55\degree~and
% and inclination of 25\degree~(left), 50\degree~(middle), and 75\degree.
% The bottom row of models have an inner warp with a fixed inclination of 80\degree~
% (representative of that in the HD~100546 disk) and a position angle of 25\degree~
% (left), 55\degree~(middle), and 85\degree~(right).
% The transition radius in all models is 100~au.
% The contours are in units of a single spectral resolution element (0.21 \kms).}
% \label{figurea3}
% \end{figure*}
\end{document}

% Figure and table environments

%\begin{deluxetable}{}
%\tablecaption{}
%\tablewidth{0pt}
%\tablehead{\colhead}
%\startdata
%\multicolumn{}{}{} \\
%\hline
% & \\
%\enddata
%\end{deluxetable}

%\begin{figure*}
%\includegraphics[width=\textwidth]{}
%\caption{}
%\label{figure1}
%\end{figure*}

%\begin{deluxetable*}{ccccc}
%\tablecaption{}
%\tablehead{\colhead{}}
%\startdata
% &  &  &  &  \\
%\enddata
%\end{deluxetable*}

% Appendix

%The line-of-sight positional offset of the top cone relative to the disk midplane is given by
%\begin{equation}
%\delta{_\mathrm{top}}(x,y,z) = \frac{r_\mathrm{mid}\tan{\phi}}{1-\tan{\phi}\tan{i}\cos{\theta}},
%\end{equation}
%with a similar expression for the bottom cone,
%\begin{equation}
%\delta{_\mathrm{bot}}(x,y,z) = \frac{r_\mathrm{mid}\tan{\phi}}{1+\tan{\phi}\tan{i}\cos{\theta}},
%\end{equation}
%where $r_\mathrm{mid} = \sqrt{x^2+y^2+z_\mathrm{mid}^2}$, $z_\mathrm{mid} = y\tan{i}$, and
%$i$ and $\phi$ have been defined previously.
%The derivation of these expressions is given in the Appendix.
%The radius of the top and bottom cone is then determined using,
%\begin{equation}
%r_\mathrm{top/bot}(x,y,z) = \sqrt{x^2 + y^2 + (z_\mathrm{mid} + \delta_\mathrm{top/bot})^2}.
%\end{equation}
%The line-of-sight velocity can then be mapped back to the sky plane $(x',y')$ using
%Equation \ref{velocity} and $r_\mathrm{top/bot}$ in the place of $r$ for the top and
%bottom cones.

%Assuming that the minimum orbital period of the warp is the time span
%between the two APEX datasets ($\approx 6$ years), then the line
%profile will cycle between blue-symmetric-red-symmetric-blue on a timescale of
%$\approx 1.5$ years.
%This means that, at the time that the ALMA data were taken,
%the line profile was expected to be in its ``symmetric" phase, provided that
%the earliest APEX data was taken in the middle of, or towards the end
%of, the line profile's ``blue" phase.
%If the above holds, this suggests the presence of a planetary companion
%in a misaligned orbit at $\approx 3$~au.
