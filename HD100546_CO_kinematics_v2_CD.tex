\documentclass[onecolumn]{aastex6}

\newcommand{\vdag}{(v)^\dagger}
\newcommand{\myemaila}{c.walsh1@leeds.ac.uk}
\newcommand{\myemailb}{cwalsh@strw.leidenuniv.nl}

%% Put in packages here
\usepackage{amsmath}
\usepackage{amssymb}
\usepackage{url}
\usepackage{gensymb}
\usepackage{graphics}
\usepackage{subfigure}
\usepackage{wasysym}
\usepackage{textcomp}
\usepackage{xcolor}
\usepackage{multirow}
\usepackage[version=3]{mhchem}
%\usepackage{multicol}
%% End of packages


\newcommand{\kms}{km~s$^{-1}$}
\newcommand{\perbeam}{beam$^{-1}$}

\shorttitle{Kinematics of the HD 100546 disk}
\shortauthors{Catherine Walsh et al.}

%% Begin document
\begin{document}

\title{NON-KEPLERIAN MOLECULAR GAS MOTIONS IN THE INNER ($\lesssim 100$~AU) DISK AROUND HD~100546}
\author{Catherine Walsh\altaffilmark{1,2}, 
Cail Daley\altaffilmark{3}, 
Stefano Facchini\altaffilmark{4}, \& 
Attila J\'{u}hasz \altaffilmark{5}}

\altaffiltext{1}{School of Physics and Astronomy, University of Leeds, Leeds, LS2 9JT, UK}
\altaffiltext{2}{Leiden Observatory, Leiden University, P.~O.~Box 9531, 2300~RA Leiden, The Netherlands}
\altaffiltext{3}{Astronomy Department, Wesleyan University, 96 Foss Hill Drive, Middletown, CT 06459, USA}
\altaffiltext{4}{Max-Planck-Institut f\"{u}r Extraterrestriche Physik, Giessenbackstrasse 1, 85748 Garching, Germany}
\altaffiltext{5}{Institute of Astronomy, University of Cambridge, Madingley Road, Cambridge CB3 0HA, UK}

\email{\myemaila; \myemailb}

\begin{abstract} 

We present spatially resolved Atacama Large
Millimeter/submillimeter Array (ALMA) images of $^{12}$CO~$\mathrm{J}=3-2$
emission from the protoplanetary disk around the Herbig Ae star, HD~100546.  We
expand upon earlier analyses of this data and model the spatially-resolved
kinematic structure (i.e., the first moment map) of the CO emission.  Assuming a
velocity profile which prescribes a flat or flared  emitting surface in
Keplerian rotation, we uncover significant residuals with a peak of
$\approx7\delta v$, where  $\delta v =  0.21$~\kms~is the width of a single
spectral resolution element. The shape and extent of the residuals reveal the
possible  presence of a severely warped and twisted inner disk extending to at
most $90-100$~au (\textbf{it strikes me as a little awkward to give a range for
a maximum value.. what if we just say `at most 100'?}). Adapting the model to
include a misaligned inner disk with (i) an inclination almost edge-on to the
line of sight and (ii) a position angle almost orthogonal to that of the outer
disk reduces the residuals to $< 3\delta v$.  HD~100546 joins several other
protoplanetary disks that exhibit  non-Keplerian kinematics as evidenced by
molecular emission in the inner region. However, our findings are contrasted by
recent VLT/SPHERE, MagAO/GPI, and VLTI/PIONIER  observations of HD~100546 that
show no evidence of a severely misaligned inner {\em dust} disk  down to spatial
scales of $\sim 1$~au.  An alternative explanation for the observed kinematics
are fast radial flows across the inner dust cavity. Regardless of the process,
the most likely cause is the presence of an unseen planetary companion. 

\end{abstract}

\section{INTRODUCTION}
\label{introduction}

Observations of protoplanetary disks around nearby young stars offer unique 
insight into the initial conditions of planetary system formation.  
Resolved continuum observations spanning optical to cm wavelengths  
reveal the spatial distribution of dust across a range of grain sizes, 
which in turn, can highlight signposts of ongoing planet formation 
and/or as yet unseen massive companions/planets (e.g., cavities, gaps, rings, 
and spirals; see the recent reviews by 
\citealt{espaillat14}, \citealt{andrews15}, and \citealt{grady15}).  
Likewise, spectrally and spatially resolved observations of molecular 
line emission disclose the spatial distribution and excitation of various 
gas species, from which information on disk gas properties can be extracted
\citep[e.g.,][]{dutrey14,sicilia-aguilar16}.

Second only to \ce{H2} in gas-phase molecular abundance, CO is a powerful
diagnostic of various properties including the disk gas mass, radial surface
density, and temperature.  The primary isotopologue $^{12}$CO is optically
thick, and thus emits  from the warm disk atmosphere; this allows the
gas temperature in this region to be derived. \citep[e.g.,][]{williams11,dutrey14}.   The
rarer isotopologues ($^{13}$CO, C$^{18}$O, C$^{17}$O and $^{13}$C$^{18}$O)  have
progressively lower opacities and so enable penetration  towards and into the
disk midplane  (see, e.g., recent theoretical studies by \citealt{bruderer13},
\citealt{miotello16}, and \citealt{yu16}).  In observations with sufficiently
high spatial resolution, now routine with ALMA,  this allows a direct
determination of the location of the CO snowline  with high precision
\citep[see, e.g.,][]{nomura16,schwarz16,zhang17}.   However, it has been
demonstrated recently that chemistry, in particular  isotope-selective
photodissociation \citep{visser09}, can complicate  the extraction of disk gas
masses from CO isotopologue emission  \citep{miotello14,miotello16}. Chemical
conversion of CO to a less volatile form, e.g., \ce{CO2},  complex organic
molecules, or hydrocarbons,   is an alternative explanation for apparently low
disk masses derived from  CO observations
\citep{helling14,furuya14,eistrup16,yu17}.  

Because emission from $^{12}$CO (and often $^{13}$CO) at (sub)mm wavelengths 
is bright, it has historically been used as a tracer of disk kinematics allowing 
a dynamical determination of the mass of the central star \citep[e.g.,][]{simon00}.  
However, gas motion can deviate from that expected due to 
Keplerian rotation alone because of a variety of different physical effects 
that can be inferred from spatially-resolved observations. 
These include spiral density waves, a substantial (and thus measurable) 
gas pressure gradient, radial flows mediated by accreting planets 
across cavities, or a disk warp 
\citep[see, e.g.,][]{rosenfeld12,tang12,casassus13,rosenfeld14,christiaens14,casassus15}.  
Spirals, radial flows, and warps can all signify the presence 
of (potentially massive) planetary companions; 
hence, perturbations from Keplerian motion traced in 
bright and spectrally- and spatially-resolved CO emission 
may expose unseen planets.  

Here, we present high signal-to-noise and spectrally-resolved 
ALMA Cycle 0 images of $^{12}$CO $\mathrm{J}=3-2$ emission from the protoplanetary 
disk around the nearby Herbig~Ae star, HD~100546.  
The HD~100546 disk has been proposed to host (at least) two 
massive companions \citep[see, e.g.,][]{acke06,quanz13,walsh14}. 
However, recent MagAO/GPI observations presented in \citet{rameau17} 
have raised doubt on the previous identification of the point source at 50~au 
as a (proto)planet by two previous and independent groups \citep{currie15,quanz15}.
In \citet{walsh14}, henceforth referred to as Paper I, 
we presented the $^{12}$CO ($\mathrm{J}=3-2$, $\nu=345.796$~GHz) first moment map 
and dust continuum emission (at 302 and 346~GHz).  
These data spatially resolved the CO emission and allowed direct 
determination of the radial extent of the molecular disk ($\approx 390$~au, 
see also \citealt{pineda14}). 
The continuum data analysed in Paper I showed that the (sub)mm-sized dust grains 
had been sculpted into two rings. 
\citet{pinilla15} showed that this 
dust morphology is consistent with dust trapping by two massive companions: 
one with mass $\approx 20 M_\mathrm{J}$ at 10~au, and one with mass 
$\approx 15 M_\mathrm{J}$ at 70~au.  
Hence, the ALMA data support the presence of an outer (proto)planet. 
Emission from $^{12}$CO ($\mathrm{J}=3-2$, $6-5$, and $7-6$) from HD~100546 
had been detected previously in single-dish observations with APEX \citep{panic10}.    
The APEX data revealed an asymmetry in the red and blue peaks 
in the double-peaked line profiles most apparent in the $\mathrm{J}=3-2$ 
and $6-5$ transitions. 
\citet{panic10} hypothesised that the asymmetry may arise 
due to shadowing of the outer disk by a warp in the inner disk.  
Using the same data set as here, \citet{pineda14} showed that the 
position-velocity ($p-v$) diagram across the major axis of the disk is 
better described by a disk inclination of $\approx 30\degree$, 
rather than an inclination of $44\degree$ that best reproduces the 
aspect ratio of the disk as seen in continuum emission (Paper I).  

In this work we revisit the HD~100546 ALMA Cycle~0 data and conduct a deeper analysis 
of the spatially and spectrally resolved $^{12}$CO $\mathrm{J}=3-2$ emission.  
The focus of this work is the search for evidence of a warp in the inner 
regions of the disk, as suggested by the single dish data presented in 
\citet{panic10}. 
In Section~\ref{almaimages}, we outline the imaging presented 
in the paper, and in 
Section~\ref{modelkinematics} we describe the modelling techniques 
used and present the results.  
Sections~\ref{discussion} and \ref{conclusion} discuss the implications 
and state the conclusions, respectively.  

\section{ALMA~IMAGING~OF~HD~100546 }
\label{almaimages}

HD~100546 was observed with ALMA on 2012 November 24 with 
24 antennas in a compact configuration, with baselines 
ranging from 21 to 375~m.  
The self-calibrated and phase-corrected 
measurement set, produced as described in Paper I, 
is used in these analyses.  
In this work, we adopt the revised distance to 
HD~100546 determined by Gaia \citep[$109\pm4$ pc,][]{gaia16a,gaia16b}, 
and a stellar mass of $2.4 \mathrm{M}_{\odot}$, \citep{vandenancker98}

In Paper I, the integrated intensity
and first moment maps from the $^{12}$CO $\mathrm{J}=3-2$ rotational 
transition at 345.795~GHz ($E_\mathrm{up} = 33.19~\mathrm{K}$ 
and $A_\mathrm{ul} = 2.497\times 10^{-6}~\mathrm{s}^{-1}$) were presented.  
The data cube from which those maps were produced was itself produced 
using the CASA task \texttt{clean} with Briggs weighting (robust=0.5) 
at a spectral resolution of 0.15~\kms.  
The resulting channel maps had an rms noise of 19 mJy~\perbeam~channel$^{-1}$ 
and a synthesised beam of $0\farcs95\times0\farcs42~(38\degree)$.  
The $^{12}$CO was strongly detected with a signal-to-noise ratio 
(S/N) of 163 in the channel maps.   

Because of the high S/N the imaging is redone here 
using uniform weighting which results in a smaller beam (and improved  
spatial resolution) at the expense of sensitivity.  
The resulting channel maps have an rms noise of 26 
mJy~beam$^{-1}$~channel$^{-1}$, a S/N of 106, and 
a synthesised beam of $0\farcs92\times0\farcs38~(37\degree)$.   
Figure~\ref{figure1} presents the channel maps.  
Emission is detected ($\ge 3\sigma$) across 111 channels: the central channel 
is centred at the source velocity of 5.7~\kms~as constrained 
previously by these data (see Paper I).  
The highest velocity emission detected is $\pm 8$~\kms relative 
to the source velocity.
Given that the disk inclination (as constrained by the outer disk) 
is 44\degree~and that the stellar mass is $2.4 \mathrm{M}_\odot$, 
emission is detected down to a radius of 15~au from the central star. 
Using the estimate of $\approx 30\degree$ for the inclination of the 
inner disk from \citet{pineda14}, reduces this radius to 8~au.  

The channel maps in Figure~\ref{figure1} reveal the classic ``butterfly'' morphology of 
spectrally- and spatially-resolved line emission from an inclined 
and rotating protoplanetary disk \citep[see, e.g.,][]{semenov08}.  
Compared with the resolved $^{12}$CO emission from 
the disk around the Herbig Ae star HD~163296 which has a similar inclination, 
there is no evidence of emission from the back side of disk that is a 
signature of CO freezeout in the disk midplane coupled with 
emission from a flared surface \citep{degregorio13,rosenfeld13}.
The blue-shifted emission also appears mostly symmetric about the disk major axes  
(especially that from the south-east of the disk) 
indicating that it arises from a relatively ``flat'' surface.   
This is in contrast with emission from the disk around the Group I 
Herbig Ae disk HD~97048 \citep{walsh16,vanderplas17}.  
However, the emission is not wholly symmetric about the disk {\em minor} axis, 
with the red-shifted emission from the north-west quadrant appearing both 
fainter, and with a positional offset, relative to blue-shifted 
emission at the same velocity.  
In Figure~\ref{figure2} the channel maps from $\pm 0.45$ to $\pm 1.5$~\kms are 
shown, now rotated counter-clockwise by 34\degree~($180\degree - $P.A.) to align the disk major axis in the vertical direction, and 
mirrored across the disk minor axis.  
Exhibiting the data in this velocity range and in this manner highlights 
the described asymmetry in brightness across the disk minor axis, the flatness of the emission, and 
the positional offset of the red-shifted north-west lobe relative to 
its blue-shifted counterpart.  
The brightest lobe to the north east, an apparent CO ``hot spot'', is 
consistent in position angle with a proposed (proto)planetry companion 
seen in direct imaging \citep{quanz13,currie15,quanz15,rameau17}.  

% \begin{figure*}[!t]
% \includegraphics[width=\textwidth]{./HD100546_CO_3-2_channel_maps.eps}
% \caption{Channel maps of the CO $\mathrm{J}=3-2$ line emission imaged at a 
% velocity resolution of 0.15~\kms. 
% Note that this is slightly over-sampled with respect to the native spectral 
% resolution of the data (0.21~\kms). 
% The dashed lines 
% represent the disk major and minor axes determined from analysis of the 
% continuum \citep{walsh14}.} 
% \label{figure1}
% \end{figure*}
% 
% \begin{figure*}[!t]
% \includegraphics[width=\textwidth]{./HD100546_CO_3-2_mirror_maps.eps}
% \caption{Channel maps of the CO $\mathrm{J}=3-2$ line emission rotated in the counter-clockwise direction 
% by 34\degree~to align the disk major axis in the vertical direction, and mirrored across the disk minor axis 
% to highlight the asymmetry in emission across the disk minor axis (now orientated in the horizontal direction).} 
% \label{figure2}
% \end{figure*}

Figure~\ref{figure3} presents the moment maps 
(zeroth, first, second, and eighth).  
The zeroth moment map (integrated intensity) was produced using 
a $3\sigma$ rms noise clip.  
The first (intensity-weighted velocity), second (intensity-weighted velocity dispersion), 
and eighth (peak flux density) moment maps were produced using a more 
conservative clip of $6\sigma$.  

The integrated intensity appears relatively symmetric about the 
disk minor axis; however, the $^{12}$CO integrated emission extends 
further to the south-west than it does to the north-east.  
This asymmetry is also evident in the eighth-moment map with the 
north-east side of the disk appearing brighter than the south-west side 
with a dark lane in the east-west direction also seen in scattered 
light images with VLT/SPHERE \citep{garufi16}. 
Both maps hint at emission from a flared disk which would lead to an 
asymmetry in integrated emission across the disk major axis (i.e., the axis of inclination).   
The first and second moment maps also hint at asymmetric emission, in particular, 
the emission at the source velocity through the inner disk is twisted relative to the 
disk minor axis determined from the continuum emission.  
The velocity dispersion in the inner disk is also not wholly symmetric across 
the disk minor axis.  
A ``by-eye'' inspection of the first- and eighth-moment maps, 
in particular, suggest the possible presence of a warp in the inner disk.  

% \begin{figure*}[!ht]
% \centering
% \subfigure{\includegraphics[width=0.45\textwidth]{./HD100546_CO_3-2_line_mom0.eps}}
% \subfigure{\includegraphics[width=0.45\textwidth]{./HD100546_CO_3-2_line_mom1.eps}}
% \subfigure{\includegraphics[width=0.45\textwidth]{./HD100546_CO_3-2_line_mom2.eps}}
% \subfigure{\includegraphics[width=0.45\textwidth]{./HD100546_CO_3-2_line_mom8.eps}}
% \caption{Moment maps for CO $\mathrm{J}=3-2$ line emission from HD 100546.  Clockwise from  top 
% left: zeroth moment map (integrated intensity, Jy~\perbeam~\kms), first moment map (intensity-weighted 
% velocity, \kms), eighth moment map (peak intensity, Jy~\perbeam), and second moment map 
% (intensity-weighted velocity dispersion, \kms).  
% The dashed contour in the second and eighth moment maps corresponds to the $3\sigma$ contour 
% of the integrated intensity.} 
% \label{figure3}
% \end{figure*}

\section{MODELLING THE KINEMATICS}
\label{modelkinematics}

Modelling of the kinematics as traced by the $^{12}$CO emission is conducted 
using analytical models which describe the line-of-sight projected velocity, 
i.e. the first moment map (\textbf{isn't the first moment map technically the
intensity-weighted projected velocity?}).  
The model moment maps are convolved with the synthesised beam of the observations.  
The residuals ($\mathrm{data} - \mathrm{model}$) are in units of a single spectral 
resolution element ($\delta v = 0.21$~\kms).
This discretisation is necessary because features smaller than the native spectral 
resolution of the data cannot be fit. 
Note that this analysis will not address the asymmetry in CO brightness across 
the disk seen in the channel maps and zeroth and eighth moment maps.  
We leave such an analysis to future work when data for multiple 
CO transitions and isotopologues allow a robust extraction of 
the gas surface density and gas temperature.  

We explore several metrics of ``best fit'':
(i) the total number of pixels for which the analytical and smoothed 
projected velocity reproduces the data within one spectral resolution element,
(ii) the sum of the square of the residuals normalised by the total number of unmasked pixels, and 
(iii) the magnitude of the peak residual. 
The total number of unmasked pixels in the observed first moment map is 2562.  
Given the relatively small number of parameters for each model considered, 
the modelling approach is grid based, i.e. all possible grid combinations are explored.  

\subsection{A flat emitting surface}

The simplest prescription for describing the first moment map of spectrally-resolved 
line emission from a disk 
is axisymmetric emission arising from a geometrically flat surface inclined to the line of sight.  
Assuming that the position angle of the disk is aligned with the 
y axis, the projected velocity on the sky relative to the observer is described by
\begin{equation}
v(x',y') = \sqrt{\frac{G M_\star}{\rho}} \sin i \sin \theta, 
\label{velocity}
\end{equation}
where $G$ is the gravitational constant, $M_\star$ is the mass of the central 
star, $\rho=\sqrt{x^2+y^2}$ is the radius, $i$ is the inclination, and 
$\theta = \arctan(y/x)$ \citep[e.g.,][]{rosenfeld13}.  
In this projection and for this particular orientation, 
$x = x'/\cos i$, $y = y'$, and $z = 0$. 
Model first moment maps for a flat disk with the same P.A as HD~100546 and
inclinations of 30\degree, 45\degree, and  60\degree\ are shown in
Figure~\ref{figurea1} in the Appendix.  

The wide range of disk inclinations ($[20\degree,60\degree]$) and disk  position
angles ($[120\degree,170\degree]$) explored are motivated by previous analyses
of the continuum data which suggested a P.A.~of $146\degree \pm 4\degree$   and
an inclination of $44\degree \pm 3\degree$ \citep[see Paper I and][]{pineda14}.
Using the CO data (\textbf{I'm a little confused--which CO data is being
referred to here? The data used in this paper?}), \citet{pineda14} suggest that
the inner disk may be better described  with an inclination of $\approx
30\degree$; as such, we extend our explored range  accordingly to ensure good coverage
over the parameter space.   First, a coarse grid with a resolution of
5\degree~is run over the full parameter space,  followed by a zoomed in region
with a resolution of 1\degree.

The top-left panel of Figure~\ref{figure4} presents a 3D plot showing the 
total number of pixels which fit the data velocity field within one spectral resolution 
element, $\delta v$, as a function of disk inclination and position angle. 
The distribution is strongly peaked: the best-fit flat disk model using this 
metric has an inclination of 36\degree~and a P.A.~of 145\degree~with 62.1\% of model pixels 
lying within one spectral resolution element of the data.  
These data are also listed in Table~\ref{table1}.  
The P.A.~is in excellent agreement with that derived from the continuum observations.  
The inclination, on the other hand, is lower and closer 
to the suggested inclination from \citet{pineda14}.

The left hand plots of Figure~\ref{figure5} show the distribution of residuals
summed  over the entire disk (top panel) and the residual first moment map
(bottom panel).   The histogram of residuals shows small dispersion of about 0
with 96.0\% of pixels  matching the data within $\pm0.315$ \kms.   The residual
map shows that a flat disk well reproduces the large-scale  velocity field: the
largest deviations from this model occur in the innermost  disk where the model
velocity field over-predicts (by up to $\approx 7\delta v$) the magnitude of the
projected line-of-sight velocity along the minor axis of the disk.   This leads
to negative residuals in the north-east  and positive residuals in the
south-west.   The morphology of the residuals suggests that the inner disk has
an  additional inclination along the minor axis of the outer disk, i.e.,  close
to orthogonal to that of the outer disk.  

% \begin{figure*}[]
% \subfigure{\includegraphics[width=0.33\textwidth]{./flat_stats.eps}}
% \subfigure{\includegraphics[width=0.33\textwidth]{./flared_lower_stats.eps}}
% \subfigure{\includegraphics[width=0.33\textwidth]{./flared_upper_stats.eps}}
% \caption{Distribution of model best-fit values using metric (i) as a function of 
% inclination and position angle for the best-fit flat disk, flared disk (lower cone), 
% and flared disk (upper cone), respectively. 
% The best-fit opening angles, $\alpha$, of the flared disks (with respect to the 
% disk midplane) are 13\degree~and 9\degree~for the lower and upper cones, respectively. 
% The percentage scale corresponds to the full range of pixel values (from 0 to 2652).} 
% \label{figure4}
% \end{figure*}
% 
% \begin{figure*}[]
% \subfigure{\includegraphics[width=0.33\textwidth]{./upper_hist_i36_p145_a00.eps}}
% \subfigure{\includegraphics[width=0.33\textwidth]{./lower_hist_i38_p142_a13.eps}}
% \subfigure{\includegraphics[width=0.33\textwidth]{./upper_hist_i36_p145_a09.eps}}
% \subfigure{\includegraphics[width=0.33\textwidth]{./upper_res_i36_p145_a00.eps}}
% \subfigure{\includegraphics[width=0.33\textwidth]{./lower_res_i38_p142_a13.eps}}
% \subfigure{\includegraphics[width=0.33\textwidth]{./upper_res_i36_p145_a09.eps}}
% \caption{Residual histograms (top) and maps (bottom) 
% using metric (i) as the measure of best fit for a geometrically flat disk (left), 
% and the lower cone (middle) and upper cone (right) of a flared disk.  
% The histograms are displayed on a log scale to emphasise the largest residuals.} 
% \label{figure5}
% \end{figure*}

\subsection{A flared emitting surface}

Although a geometrically flat disk well reproduces much of  the velocity field,
particularly for the outer disk,  we test next whether emission from a flared
surface can improve upon the flat disk fit.   This is important to check because
HD~100546 is classified as a Group I (i.e., flared)  Herbig Ae star
\citep{meeus01}, so one might expect the $^{12}$CO emission to arise from a
layer  higher up in the disk atmosphere.   Indeed, thermo-chemical modelling of
the disk around HD~100546 by \citet{bruderer12}  suggests that the $^{12}$CO
line emission arises from a layer $z/\rho \approx 0.2$. 

\citet{rosenfeld13} modelled the emission from the disk around  the Herbig~Ae
star HD~163296 by assuming that the emission arises from  an inclined and flared
surface with some opening angle, $\alpha$, relative  to the $(x,y)$ plane (the
disk midplane), i.e., a ``double-cone'' morphology.     In this way, the front
and back sides of the disk with the same projected  line-of-sight velocity are
spatially offset \citep[see also][]{degregorio13}.   Here, a similar toy model
is used; however, to determine the  line-of-sight velocity, the radius is
defined using spherical coordinates ($r=\sqrt{x^2+y^2+z^2}$) rather than 
cylindrical coordinates  ($\rho=\sqrt{x^2+y^2}$; \citealt{rosenfeld13}).   For
small opening angles the two methods give similar results:  the radii differ by
no more than 10\% for $\alpha \le 25\degree$.   A flared disk with this emission
morphology has two possible orientations, with either the lower or the upper
face of the cone visible to the observer (see e.g., figure 3 in
\citealt{rosenfeld13}).   Model first moment maps for a flared disk with the
same P.A and inclination as  HD~100546 but with different opening angles
(20\degree, 45\degree, and  60\degree) are shown in Figure~\ref{figurea2} in the
Appendix.  

The range of surface opening angles ($[0\degree,20\degree]$) is motivated by
previous thermo-chemical  modelling of CO emission from HD~100546  which suggest
an opening angle $\alpha \approx 11\degree$ \citep[][]{bruderer12}.   The
symmetry in the channel maps (Figure~\ref{figure3}) also suggests that the
emitting layer  lies relatively close to the midplane.   As before, a coarse
grid with a resolution of 5\degree~is initially  run over the full parameter
space, followed by a zoomed in grid  with a resolution of 1\degree.

Figures~\ref{figure4} and \ref{figure5} present the statistics and residuals 
for the best-fit lower cone and upper cone of a flared disk.  
Using metric (i), the best-fit upper cone model fits the data marginally better 
(reproducing 65.0\% of the velocity field) than both the flat disk and the best-fit lower 
cone model (62.1\% and 62.6\%, respectively, see Table~\ref{table1}).  
The best-fit inclination, P.A., and opening angle are 36\degree, 
145\degree, and 9\degree, respectively (see Table~\ref{table1}).  
The opening angle of the $^{12}$CO-emitting surface agrees well with that 
suggested by thermo-chemical models of HD~100546 \citep{bruderer12}.

The best-fit lower cone model has an inclination of 38\degree,  a P.A.~of
142\degree, and an opening angle of 13\degree.   The inclination of this model
lies closest to that derived from the  continuum observations ($44\degree \pm
3\degree$).   Despite resulting in a marginally worse fit to the data than  the
upper cone model (see Table~\ref{table1}), a ``by-eye''  examination of the
residual map (bottom left panel of  Figure~\ref{figure5}) shows that this
morphology best reproduces the velocity field in all quadrants  of the outer
disk excepting the north-west quadrant for  which the magnitude of the velocity
field is over-estimated.   Comparing this residual map to both the  channel map
(Figure~\ref{figure1}) and the  eighth moment map (bottom right panel of
Figure~\ref{figure3})  shows that emission from this quadrant appears less
bright  and exhibits a positional offset relative to  that mirrored across the
minor axis of the disk.   However, that the upper cone model fits the data best
using this metric is in agreement with recent VLT/SHERE images that confirm that
the  the far side of the flared disk surface lies towards the north
\citep{garufi16}.  

The global best-fit across all three models as determined by metric (ii), i.e.,
sum of the squares of the residuals normalised by the total number of pixels,
is a flat disk with an inclination of 37\degree~and a P.A.~of 142\degree.   The
best-fit model selected by the smallest peak residual,  i.e., metric (iii), is
also shared by all three models and  is a flat disk with an inclination of
39\degree~(again in good agreement  with the other two metrics); however, the
disk P.A which gives the  smallest peak residual is 126\degree.   The residual
histograms and maps for both of these models are  shown in Figure~\ref{figure6}.
That the inner disk velocity structure is better fit with a shallower  P.A.~than
the outer disk, highlights the presence of a twisted warp:  this is investigated
in the subsequent section. 

\begin{deluxetable*}{lcccccccc}
\tablecaption{Best-fit parameters for the flat and flared kinematic models. \label{table1}}
\tablewidth{0pt}
\tablehead{\colhead{Model} & \colhead{Metric}  & \colhead {Inclination} & \colhead{P.A.} & \colhead{Opening} 
& \colhead{Pixel} & \colhead{Percentage}  & \colhead{Sum of} & \colhead{Peak} \\
  & \colhead{of best-fit} &  &  & \colhead{angle} 
& \colhead{number} &    & \colhead{residual squares} & \colhead{residual}}
\startdata
\hline
Flat                & (i)   & 36\degree & 145\degree & 0\degree  & 1590 & 62.1\%  & 0.737 & 7.23 \\
                    & (ii)  & 37\degree & 142\degree & 0\degree  & 1550 & 60.5\%  & 0.665 & 6.71 \\
                    & (iii) & 39\degree & 126\degree & 0\degree  &  224 &  8.8\%  & 3.610 & 3.63 \\
Flared (lower cone) & (i)   & 38\degree & 142\degree & 13\degree & 1605 & 62.6\%  & 0.718 & 6.64 \\
Flared (upper cone) & (i)   & 36\degree & 145\degree & 9\degree  & 1665 & 65.0\%  & 0.764 & 7.28 \\
\enddata
\end{deluxetable*}

% \begin{figure*}[]
% \centering
% \subfigure{\includegraphics[width=0.33\textwidth]{./upper_hist_i37_p142_a00.eps}}
% \subfigure{\includegraphics[width=0.33\textwidth]{./upper_hist_i39_p126_a00.eps}}
% \subfigure{\includegraphics[width=0.33\textwidth]{./upper_res_i37_p142_a00.eps}}
% \subfigure{\includegraphics[width=0.33\textwidth]{./upper_res_i39_p126_a00.eps}}
% \caption{Residual histograms (top) and maps (bottom) for a geometrically flat, rotating disk, 
% using metrics (ii) and (iii) as the measure of best fit.  
% The histograms are displayed on a log scale to emphasise the largest residuals.} 
% \label{figure6}
% \end{figure*}

\subsection{A warped disk}

The residual maps displayed in Figures~\ref{figure5} and \ref{figure6} 
reveal two features: (i) a rotating disk within $\approx$1\farcs0 of the source 
position with an inclination angle approximately orthogonal to the line of sight, and 
(ii) a shallower position angle on small scales ($\lesssim$1\farcs0) than on larger scales.  
Both results point towards a twisted warp in the inner disk 
\citep[see, e.g,][and references therein]{juhasz17}.

Because the residuals are of the order of the size of the synthesised beam, 
a simple toy prescription for the warp is used.  
The inner disk is modelled as a planar disk within a fixed radius which 
possesses its own inclination and P.A, i.e., the inner disk is 
misaligned relative to the outer disk.  
This is similar to the approach used by \citet{rosenfeld14} to 
model the kinematics of HD~142527.  
Figure~\ref{figurea3} presents model first moment maps for a warped 
disk for a range of inclinations and position angles; the 
transition radius is fixed at 100~au, and the outer disk parameters 
are given values appropriate for the HD~100546 disk.

The outer disk velocity structure is fixed to that of the best-fit upper cone model. 
As mentioned in the previous section, recent VLT/SPHERE images of scattered light 
from HD~100546 suggest that the far side of the (flared) disk lies towards the 
north-east \citep{garufi16}.  
This results in three additional fitting parameters only: 
the inner disk inclination ([40\degree, 90\degree]), the inner disk P.A.~([40\degree, 100\degree]), and 
a transition radius marking the boundary between the inner 
and outer disks ([40, 120]~au).  
Note that, for simplicity, we assume that the inner disk velocity structure is 
described using the flat disk prescription (i.e., Equation~\ref{velocity}). 
A coarse grid with a resolution of 10\degree~and 10~au is first run to 
identify the parameter space containing the global best-fit, followed by 
a finer grid over this zoomed-in region (with a resolution of 2\degree~and 2~au).  

Figures~\ref{figure7} and \ref{figure8} present the statistics and residuals for
the best-fit warped disk, respectively.   Metric (i) favours a model with an
inner disk that is almost ``edge-on''  ($i=80\degree$) to the line of sight,
almost orthogonal to the outer disk  major axis (P.A.$= 60\degree$), and with a
transition radius of 90~au  (see Figure~\ref{figure7} and Table~\ref{table2}).
These values are consistent with the morphology of the residuals of both  the
flat and flared models (see Figures~\ref{figure5} and \ref{figure6}).   The
magnitude of the peak residual of this model is significantly smaller than  the
previous models selected using metric (i), $4 \delta v$ versus $7 \delta v$.
Metrics (ii) and (iii) select the same model (see Table~\ref{table2})  with
parameters similar to those using metric (i); an inclination of  84\degree, a
P.A. of 64\degree, and a transition radius of 100~au.   Comparing the residual
histograms and maps for these two models (shown in  Figure~\ref{figure8}),
highlights how a small change in inclination and/or  position angle can
significantly reduce the magnitude of the peak residual.   This latter model
results in a peak residual of only $2.4 \delta v$ and has  the smallest
dispersion of residuals: 98\% of pixels match the data within  $\pm
0.315$~\kms~and 100\% of pixels match within $\pm 0.525$ \kms. 

% \begin{figure}[]
% \centering
% \subfigure{\includegraphics[width=0.33\textwidth]{./warp_stats.eps}}
% \caption{Distribution of model best-fit values using metric (i) as a function of 
% inclination and position angle for the best-fit warped disk.  
% The best-fit transition radius using this metric is 90~au. 
% In this plot, the percentage scale corresponds to the z-axis range.} 
% \label{figure7}
% \end{figure}
% 
% \begin{figure*}[]
% \centering
% \subfigure{\includegraphics[width=0.33\textwidth]{./warp_hist_r090_i080_p060.eps}}
% \subfigure{\includegraphics[width=0.33\textwidth]{./warp_hist_r100_i084_p064.eps}}
% \subfigure{\includegraphics[width=0.33\textwidth]{./warp_res_r090_i080_p060.eps}}
% \subfigure{\includegraphics[width=0.33\textwidth]{./warp_res_r100_i084_p064.eps}}
% \caption{Residual histograms (top) and maps (bottom) for a protoplanetary disk 
% with a warped inner disk using metrics (i) and (ii) as the metric of best fit.  
% The histograms are displayed on a log scale to emphasise the largest residuals. 
% Note that metrics (ii) and (iii) select the same warped inner disk parameters 
% (see Table 2).} 
% \label{figure8}
% \end{figure*}

\begin{deluxetable*}{lcccccccc}
\tablecaption{Best-fit parameters for the warped kinematic models. \label{table2}}
\tablewidth{0pt}
\tablehead{\colhead{Model} & \colhead{Metric}  & \colhead {Inclination} & \colhead{P.A.} & \colhead{Transition} 
& \colhead{Pixel} & \colhead{Percentage}  & \colhead{Sum of} & \colhead{Peak} \\
  & \colhead{of best-fit} &  &  & \colhead{radius (au)} 
& \colhead{number} &    & \colhead{residual squares} & \colhead{residual}}
\startdata
\hline
Warped              & (i)            &  80\degree &  60\degree &  90 & 1722  & 67.2\%  & 0.387 & 4.064 \\
                    & (ii) \& (iii)  &  84\degree &  64\degree & 100 & 1710  & 66.4\%  & 0.350 & 2.441 \\
\enddata
\end{deluxetable*}

\section{DISCUSSION}
\label{discussion}

The analysis of the kinematics of the HD~100546 protoplanetary disk  presented
here has indicated the  possible presence of a severely warped (misaligned)
inner disk within 100~au of the central star.   The inner gas disk exhibits a
position angle that is almost orthogonal to that of the outer disk,  and is
inclined almost edge-on to the line-of-sight.   Figure~\ref{figure9} shows an
idealised model of the HD~100546  protoplanetary disk.   The morphology of the
intermediate region between the inner and outer disks is not yet known, and so
is intentionally left blank in the cartoon.   Higher spatial resolution data are
needed to reveal the intermediate region morphology and  determine if the
proposed warp lies within a smaller radial region than suggested by these data.
Scattered light images taken with VLT/SPHERE and MagAO/GPI with a spatial
resolution  of 0\farcs02 and 0\farcs01, respectively, reveal no evidence of a
severely misaligned  dust disk beyond $\approx10$~au
\citep{garufi16,follette17}. However, both datasets do suggest the presence of
spirals in the inner disk within 50~au, and resolve the inner  edge of the outer
dust disk traced in small grains (11-15~au, depending on wavelength). Further,
recent VLTI/PIONIER interferometric observations of HD~100546 reported in the
survey by \citet{lazareff17}, and which have a spatial resolution of order $\sim
1$~au,  suggest that the very innermost dust disk has a similar position  angle
(152\degree) and inclination (46\degree) as the outer disk traced in  sub-mm
emission \citep[146\degree\ and 44\degree, respectively;][]{walsh14}.   Hence,
if the non-Keplerian motion of  the CO gas $\lesssim 100$~au is indeed caused by
an extreme warp, in light  of these new observations it appears that the small
dust grains may have a different morphology than the gas in the inner disk. This is
discussed further below.

The presence of a disk warp in HD~100546 was originally  proposed by
\citet{panic10} and based on asymmetries seen  in the red and blue peaks of
single-dish spectra  observed with APEX.   The ALMA data presented here show no
such asymmetries in the  peaks of the line profile (see Figure~\ref{figure10});
however,  the mirrored line profile does highlight that the red and blue  lobes
of emission have different shapes when integrated over the disk.   This is
consistent with the morphology seen in the channel maps  and described in
Section~\ref{almaimages}.    

% \begin{figure}[]
% \centering
% \includegraphics[width=0.5\textwidth]{./warp_cartoon2.pdf}
% \caption{Cartoon of the HD~100546 protoplanetary disk.  
% The colour scale indicates the $z$ coordinate of the inner and outer 
% disks relative to the sky plane ($z=0$).  
% Note that the morphology of the intermediate region between the inner and outer 
% disks is not yet known and so it is intentionally left blank.} 
% \label{figure9}
% \end{figure}
% 
% \begin{figure}[]
% \centering
% \includegraphics[width=0.5\textwidth]{./CO_346GHz_line_profile.eps}
% \caption{CO $\mathrm{J}=3-2$ line profile extracted from within the 
% $3\sigma$ contour of the integrated intensity (see Figure~\ref{figure3}).} 
% \label{figure10}
% \end{figure}

A warp in the inner disk  could shadow the outer disk and lead to an asymmetric
temperature structure,  because one side of the disk is more illuminated by the
central star  than the opposite side (see figure 3 in \citealt{panic10}).
Because the $^{12}$CO emission is optically thick, it traces  the gas
temperature, and thus any perturbations, in the emitting layer, usually  the
disk atmosphere.   The APEX $^{12}$CO $\mathrm{J}=3-2$ line profile from
\citet{panic10}  shows a stronger blue peak.   The dates of the observations
taken with  APEX and ALMA are November 2008 \citep[][]{panic10} and November
2012 \citep[][]{walsh14}, respectively.   There also exist multiple APEX
observations of  the $^{12}$CO $\mathrm{J}=6-5$ transition.   The line profile
presented in  \citet{panic10} shows an even higher-contrast asymmetry in the
blue and red  peaks than in the $\mathrm{J}=3-2$ line.   More recent
observations taken between April and July 2014 and presented in \citet{kama16}
show a similar asymmetry; however,  the ratio between the peaks is within the
flux density uncertainties  of the observations (M.~Kama, priv.~commun.). Thus,
it is currently difficult to draw concrete conclusions on any periodicity in the
shapes of the line profiles.    Conservatively assuming that the warp has
rotated $\approx 90\degree$ between  acquirement of the original APEX data (with
the stronger blue peak) and the ALMA data  (with no asymmetry in the line
profile) suggests a warp precession period of the order of $16$~yr.  

The presence of a warp indicates the presence of a perturbing companion.   A
massive ($\approx 20 M_\mathrm{J}$) planetary companion has been proposed to
reside  at $\approx 10$~au in HD~100546. This is thought to be responsible for
clearing the inner cavity in both  the dust and gas
\citep{mulders13,panic14,walsh14,pinilla15,wright15,garufi16},   as well as
inducing dynamical perturbations in [OI], OH, and CO line emission  from the
inner region \citep{acke06,vanderplas09,brittain14}.    In addition, a second
companion, which may still be in the act of formation,  has been proposed to
reside at $\approx50$~au with a mass estimate of $\approx 15 M_\mathrm{J}$
\citep{quanz13,walsh14,quanz15,currie15,pinilla15}.   However, as mentioned in
the introduction, the identification  of this point source as a (proto)planet
has been disputed in a recent  analysis of MagAO/GPI observations showing
close-to-zero proper motion of  the point source on a 4.6~yr timescale at the
$2\sigma$ level \citep{rameau17}.  Given that the presence of the outer ring of
(sub)mm dust emission is  difficult to explain in the absence of an outer
companion \citep{walsh14,pinilla15},  we proceed in remainder of the discussion
with the assumption that this disk  ``feature'' ideed arises due to a
(proto)planet.

The precession period of a warp induced by a perturbing companion can be
estimated  using equation (2) in \citet{debes17}, used for the case of TW~Hya,
and which was originally derived by \citet{lai14},   \begin{equation} P \propto
(\mu_c \cos i_c)^{-1}(M_\ast)^{-1/2}(r_\mathrm{disk})^{-3/2}(a_c)^{3}
\end{equation} where $\mu_c$ is the mass ratio of the companion to the central
star (here $2.4 \mathrm{M}_\odot$),  $i_c$ is the angle between the star-planet
orbital plane  and disk plane (here 80\degree),  $r_\mathrm{disk}$ is the radius
of the outer edge of the inner disk, and  $a_c$ is the orbital radius of the
planet. Assuming that the  protoplanet at 50~au is the perturbing body, and
setting $\mu_c = 0.006$, $r_\mathrm{disk} = $40~au, and $a_c=50$~au results in
a precession period of 1940~yr.   Hence, it is unlikely that the outer companion
has  perturbed the inner disk creating a warp that explains  the change in shape
in line profile seen in the single-dish data and these ALMA data on a 4-year
baseline.   Assuming instead that the inner companion is triggering the  warp,
and that the inner disk cavity is not completely devoid of gas  ($\mu_c =
0.008$, $r_\mathrm{disk} = $10~au, $i_c = 80$\degree, and  $a_c=10$~au), leads
to a precession period of 111~yr.  Keeping all other parameters constant, a
companion mass of  $\approx 140 \mathrm{M}_\mathrm{J}$ ($\approx 0.1
\mathrm{M}_\odot$) is required for  a warp precession period of $\approx 16$~yr.
There is no evidence that HD~100546 is in  a binary system with an M-dwarf star.
Alternatively, a closer-in companion of only  $\approx 4.5
\mathrm{M}_\mathrm{J}$ at 1~au and assuming an inner disk radius of 1~au,  leads
to a similar precession period. However, these ALMA data suggest that the
non-Keplerian gas motion lies significantly beyond  1~au since the highest
velocity gas detected only probes down to a radius of 8-15~au depending  upon
the assumed disk inclination.   Longer baseline (in time) observations are
required to determine if any  clear periodicity in the line profile shape
exists, and to then  relate this more concretely to the precession period of an
inner disk warp.  

Similar analyses have revealed warps in disks around other stars.  
\citet{rosenfeld12} showed that the kinematic structure of TW Hya 
could be reproduced using a parametric model of a disk warp 
with a moderate inclination ($\approx~8\degree$) at a radius of 5~au.  
\citet{facchini14} demonstrated the observed warp amplitude  
could be induced by a misaligned close-in companion as massive as 
$\approx 14 \mathrm{M}_\mathrm{J}$ orbiting within 4~au of the central star.  
New scattered light data from HST/STIS, coupled with 
archival data over a 17-year baseline, reveal an orbiting 
azimuthal brightness asymmetry, with a 
period of $\approx 16$~yr \citep{debes17}.  
The authors argue that this is consistent with partial 
shadowing of the outer disk by a misaligned inner 
disk interior to 1~au and that is precessing due to 
the presence of a roughly Jupiter-mass planetary companion.  

Given that the ALMA data presented here 
suggest non-Keplerian motion of the gas on spatial 
scales $\gtrsim 10$~au, it is worth considering alternative 
explanations to a warp.  
HD~142527 is another protoplanetary disk in which  
non-Keplerian motion traced by \ce{HCO+} 
($\mathrm{J}=4-3$) and $^{12}$CO ($\mathrm{J}=6-5$) 
emission has been observed \citep{casassus13,casassus15}.  
HD~142527 is a transition disk which possesses the largest known
dust cavity in sub(mm) emission 
\citep[$\approx 140$~au;][]{casassus13,fukagawa13}. 
The observed gas motions were postulated to arise from fast 
(near free-fall) radial flows across the cavity and 
accretion onto the central star via a severely misaligned inner disk 
\citep{casassus13,casassus15,marino15}.  
\citet{marino15} propose that that inner and outer disk are misaligned 
by 70\degree.  
Compared with the misalignment parameters derived for HD~142527, those 
derived here for HD~100546 appear not so extreme.  
Also, ``hot-off-the-press'' ALMA observations of 
\ce{HCO+} ($\mathrm{J}=3-2$) line emission from AA~Tau at an angular 
resolution of 0\farcs2 ($\approx 30$~au), also show evidence of non-Keplerian motions due 
to either a warp or fast radial inflow in the inner regions  \citep{loomis17}.  
Similar to here, this was revealed by a clear twist in the first 
moment map of the molecular line emission.  
However, \citet{rosenfeld14} conducted a theoretical study of 
fast radial flows in transition disks and showed that the 
derived velocity profile can also mimic that characteristic of an inner 
disk warp.  
To determine which physical mechanism is responsible for the 
observed gas motions in HD~100546 will require data at sufficient 
spatial and spectral resolution to resolve molecular emission within 
the gas cavity i.e., within a diameter of $\approx 20$~au.  

\section{CONCLUSION}
\label{conclusion}

The data and analyses shown here cleary demonstrate that 
the kinematic structure of the disk around HD~100546 cannot be 
described by a purely Keplerian velocity profile with a universal inclination 
and position angle.  
Given the current evidence for the presence of (at least) 
one massive planetary companion orbiting within 10~au of the central 
star, the presence of a disk warp, albeit with extreme parameters, 
has been our favoured explanation.  
Higher spatial resolution data are vital for corroborating 
this warp hypothesis.  
However, recent near-IR interferometric observations 
suggest that the very inner dust disk is not misaligned 
relative to the outer dust disk traced in (sub)mm emission.
This suggests that the gas and dust disks may have different morphologies 
on radial scales of $\sim 1$ to 10's of au.  
To determine whether the alternative explanation of fast radial flows could be 
responsible for the observed kinematic structure of HD~100546, higher spatial 
{\em and} spectral resolution data are required to resolve gas emission across 
the inner dust ($\lesssim 25$~au) and gas ($\lesssim 10$~au) cavities 
and explore any spatial association with the proposed planetary candidate(s). 

\begin{thebibliography}{}
\bibitem[Acke \& van den Ancker(2006)]{acke06} Acke, B. \& van de Ancker, M.~E.~2006, \aap, 449, 267
\bibitem[van den Ancker et al.(1998)]{vandenancker98} van~den~Ancker, M.~E., de~Winter, D., Tijn A Djie, H.~R.~E., \aap, 330, 145
\bibitem[Andrews(2015)]{andrews15} Andrews, S.~M. 2015, \pasp, 127, 961
\bibitem[Brittain et al.(2009)]{brittain09} Brittain, S.~D., Najita, J.~R., \& Carr, J.~S. 2009, \apj, 702, 85
\bibitem[Brittain et al.(2014)]{brittain14} Brittain, S.~D., Carr, J.~S., Najita, J.~R., Quanz, S.~P., \& Meyer, M.~R. \apj, 791, 136
\bibitem[Bruderer et al.(2012)]{bruderer12} Bruderer, S., van~Dishoeck, E.~F., Doty, S.~D., \& Herczeg, G. 2012, \aap, 541, 91
\bibitem[Bruderer(2013)]{bruderer13} Bruderer, S. 2013, \aap, 559, A46
\bibitem[Casassus et al.(2013)]{casassus13} Casassus, S., van der Plas, G., P\'{e}rez, S.~M., et al.~2013, Nature, 493, 191
\bibitem[Casassus et al.(2015)]{casassus15} Casassus, S., Marino, S., P\'{e}rex, S., et al.~2015, \apj, 811, 92
\bibitem[Christiaens et al.(2014)]{christiaens14} Christiaens, V., Casassus, S., Perez, S., van der Plas, G., \& M\'{e}nard, F. 2014, \apjl, 785,L12
\bibitem[Currie et al.(2015)]{currie15} Currie, T., Cloutier, R., Brittain, S., et al.~2015, \apjl, 814, L27
\bibitem[Debes et al.(2017)]{debes17} Debes, J.~H., Poteet, C.~A., Jang-Condell, H., et al.~2017, \apj, 835, 205
\bibitem[Dutrey et al.(2014)]{dutrey14} Dutrey, A., Semenov, D., Chapillon, E., et al.~2014, in Protostars and Planets VI, ed. H. Beuther et al.~(Tucson, AZ: Univ. Arizona Press), 317
\bibitem[Eistrup et al.(2016)]{eistrup16} Eistrup, C., Walsh, C., \& van Dishoeck, E.~F. 2016, \aap, 595, A83
\bibitem[Espaillat et al.(2014)]{espaillat14} Espaillat, C., Muzurolle, J., Najita, J., et al.~2014, in Protostars and Planets VI, ed. H. Beuther et al.~(Tucson, AZ: Univ. Arizona Press), 497
\bibitem[Facchini et al.(2014)]{facchini14} Facchini, S., Ricci, L., \& Lodato, G. 2014, \mnras, 442, 3700
\bibitem[Follette et al.(2017)]{follette17} Follette, K.~B., Rameau, J., Dong, R., et al.~2017, \aj, in press (arXiv:1704.06260)
\bibitem[Fukagawa et al.(2013)]{fukagawa13} Fukagawa, M., Tsukagoshi, T., Momose, M., et al.~2013, \pasj, 65, L14
\bibitem[Furuya \& Aikawa(2014)]{furuya14} Furuya, K. \& Aikawa, Y. 2014, \apj, 790, 97
\bibitem[Gaia Collaboration(2016a)]{gaia16a} Gaia Collaboration et al.~2016a, \aap, 595, A1
\bibitem[Gaia Collaboration(2016b)]{gaia16b} Gaia Collaboration et al.~2016b, \aap, 595, A2
\bibitem[Garufi et al.(2016)]{garufi16} Garufi, A., Quanz, S.~P., Schmid, H.~M., et al. 2016, \aap, 588, A8 
\bibitem[Grady et al.(2015)]{grady15} Grady, C., Fukagawa, M., Maruta, Y., et al.~2015, \apss, 355, 253
\bibitem[de Gregorio Monsalvo et al.(2013)]{degregorio13} de Gregorio Monsalvo, I., M\'{e}nard, F., Dent, B., et al.~2013, \aap, 557, 133
\bibitem[Helling et al.(2014)]{helling14} Helling, C., Woitke, P., Kamp, I., et al.~2014, Life, 4, 142
\bibitem[Juh\'{a}sz \& Facchini(2017)]{juhasz17} Juh\'{a}sz, A. \& Facchini, S. 2017, \mnras, in press
\bibitem[Kama et al.(2016)]{kama16} Kama, M., Bruderer, S., van~Dishoeck, E.~F., et al.2016, \aap, 592, A83
\bibitem[Lai(2014)]{lai14} Lai, D. 2014, \mnras, 440, 3532
\bibitem[Lazareff et al.(2017)]{lazareff17} Lazareff, B., Berger, J.-P., Kluska, J., et al. 2017, \aap, 599, A85
\bibitem[Loomis et al.(2017)]{loomis17} Loomis, R.~A., \"{O}berg, K.~I., Andrews, S.~M., \& MacGregor, M.~A. 2017, \apj, in press (arXiv : 1704.02006)
\bibitem[Marino et al.(2015)]{marino15} Marino, S., P\'{e}rez, S., \& Casassus, S. 2015, \apjl,  798, L44 
\bibitem[Meeus et al.(2001)]{meeus01} Meeus, G., Waters, L.~B.~F.~M., Bouwman, J., et al.~2001, \aap, 365, 476 
\bibitem[Miotello et al.(2014)]{miotello14} Miotello, A., Bruderer, S., \& van~Dishoeck, E.~D.~2014, \aap, 572, A96
\bibitem[Miotello et al.(2016)]{miotello16} Miotello, A., van~Dishoeck, E.~D., Kama, M., \& Bruderer, S.~2016, \aap, 594, A85 
\bibitem[Mulders et al.(2013)]{mulders13} Mulders, G.~D., Paardekooper, S.-J., Pani\'{c}, O., et al. 2013, 557, 68
\bibitem[Nomura et al.(2016)]{nomura16} Nomura, H., Tsukagoshi, T., Kawabe, R., et al.~2016, \apjl, 819, L7
\bibitem[Pani\'{c} et al.(2010)]{panic10} Pani\'{c}, O., van~Dishoeck, E.~F., Hogerheijde, M., et al.~2010, \aap, 519, 110
\bibitem[Pani\'{c} et al.(2014)]{panic14} Pani\'{c}, O., Ratzka, Th., Mulders, G.~D., et al. 2014, 562, A101
\bibitem[Perez et al.(2015)]{perez15} Perez, S., Dunhill, A., Casassus, S., et al.~2015, \apj, 811, L5
\bibitem[Pineda et al.(2014)]{pineda14} Pineda, J., Quanz, S.~P., Meru, F., et al.~2014, \apjl, 788, L34  
\bibitem[Pinilla et al.(2015)]{pinilla15} Pinilla, P., Birnstiel, T., \& Walsh, C.~2015, \aap, 580, A105
\bibitem[van der Plas et al.(2009)]{vanderplas09} van~der~Plas, G., van de Ancker, M. E., Acke, B., et al. 2009, \aap, 500, 1137 
\bibitem[van der Plas et al.(2017)]{vanderplas17}  van~der~Plas, G., Wright, C.~M., M\'{e}nard, F., et al.~2017, \aap, 597, A32
\bibitem[Quanz et al.(2013)]{quanz13} Quanz, S., Amara, A., Meyer, M.~R., Kenworthy, M.~A., Kasper, M., \& Girard, J.~A.~2013, \apjl, 766, L1
\bibitem[Quanz et al.(2015)]{quanz15} Quanz, S.~P., Amara, A., Meyer, M., et al.~2015, \apj, 807, 64   
\bibitem[Rameau et al.(2017)]{rameau17} Rameau, J., Follette, K.~B., Pueyo, L., et al.~2017, \aj, 153, 244
\bibitem[Rosenfeld et al.(2012)]{rosenfeld12} Rosenfeld, K.~A., Qi, C., Andrews, S.~M., et al.~2012, \apj, 757, 129
\bibitem[Rosenfeld et al.(2013)]{rosenfeld13} Rosenfeld, K.~A., Andrews, S.~M., Hughes, A.~M., Wilner, D.~J., \& Qi, C.~2012, \apj, 757, 129
\bibitem[Rosenfeld et al.(2014)]{rosenfeld14} Rosenfeld, K.~A., Chiang, E., \& Andrews, S.~M. 2014, \apj, 782, 62
\bibitem[Schwarz et al.(2016)]{schwarz16} Schwarz, K.~R., Bergin, E.~A., Cleeves, L.~I., et al.~\apj, 823, 91
\bibitem[Semenov et al.(2008)]{semenov08} Semenov, D., Pavlyuchenkov, Ya., Henning, Th., Wolf, S., \& Launhardt, R. 2008, \apjl, 673, L195  
\bibitem[Sicilia-Aguilar et al.(2016)]{sicilia-aguilar16} Sicilia-Aguilar, A., Banzatti, A., Carmona, A., et al.~2016, \pasa, 33, 59
\bibitem[Simon et al.(2000)]{simon00} Simon, M., Dutrey, A., \& Guilloteau, S. 2000, \apj, 545, 1034
\bibitem[Tang et al.(2012)]{tang12} Tang, Y.-W., Guilloteau, S., Pi\'{e}tu, V., et al. 2012, \aap, 547, A84
\bibitem[Visser et al.(2009)]{visser09} Visser, R., van~Dishoeck, E.~F., \& Black, J.~H. 2009, \aap, 503, 323
\bibitem[Williams \& Cieza(2011)]{williams11} Williams, J. P. \& Cieza, L.~A. 2011, \araa, 49, 67
\bibitem[Walsh et al.(2014)]{walsh14} Walsh, C., Juh\'{a}sz, A., Pinilla, P., et al.~2014, \apjl, 791, L6
\bibitem[Walsh et al.(2016)]{walsh16} Walsh. C., Juh\'{a}sz, A., Meeus, G., et al.~2016, \apj, 831, 200
\bibitem[Wright et a.(2015)]{wright15} Wright, C.~M., Maddison, S.~T., Wilner, D.~J., et al. 2015, \mnras, 453, 414
\bibitem[Yu et al.(2016)]{yu16} Yu, M., Willacy, K., Dodson-Robinson, S.~E., Turner, N.~J., \& Evans II, N.~J. 2016, \apj, 822, 53  
\bibitem[Yu et al.(2017)]{yu17} Yu, M., Evans II, N.~J., Dodson-Robinson, S.~E., Willacy, K., \& Turner, N.~J. 2017, \apj, in press (arXiv : 1704.05508)
\bibitem[Zhang et al.(2017)]{zhang17} Zhang, K., et al.~2017, Nature Astronomy, 
\end{thebibliography}

\appendix

\section{Model first moment maps}

Figure~\ref{figurea1} shows the first moment maps for a geometrically 
flat protoplanetary disk for a range of disk inclinations.  
Figure~\ref{figurea2} shows model first moment maps for a flared 
disk with a fixed inclination for a range opening angles, $\alpha$, of the emitting 
surface. 
Figure~\ref{figurea3} shows model first moment maps for a warped inner disk 
with a fixed transition radius of 100~au, and a range 
of warp inclinations and position angles.  
The outer disk parameters are those from the best-fit upper cone model 
(i.e., an inclination of 36\degree, a P.A.~of 145\degree, and an opening angle 
of 9\degree).

% \begin{figure*}[]
% \subfigure{\includegraphics[width=0.33\textwidth]{./upper_mom1_i30_p145_a00.eps}}
% \subfigure{\includegraphics[width=0.33\textwidth]{./upper_mom1_i45_p145_a00.eps}}
% \subfigure{\includegraphics[width=0.33\textwidth]{./upper_mom1_i60_p145_a00.eps}}
% \caption{Model first moment maps for a geometrically flat disk with the a P.A.~of 
% 145\degree~(representative of the HD~100546 disk) at an inclination of 30\degree~(left), 
% 45\degree~(middle), and 60\degree~(right).  
% The contours are in units of a single spectral resolution element (0.21 \kms).} 
% \label{figurea1}
% \end{figure*}
% 
% \begin{figure*}[]
% \subfigure{\includegraphics[width=0.33\textwidth]{./lower_mom1_i40_p145_a20.eps}}
% \subfigure{\includegraphics[width=0.33\textwidth]{./lower_mom1_i40_p145_a35.eps}}
% \subfigure{\includegraphics[width=0.33\textwidth]{./lower_mom1_i40_p145_a50.eps}}
% \subfigure{\includegraphics[width=0.33\textwidth]{./upper_mom1_i40_p145_a20.eps}}
% \subfigure{\includegraphics[width=0.33\textwidth]{./upper_mom1_i40_p145_a35.eps}}
% \subfigure{\includegraphics[width=0.33\textwidth]{./upper_mom1_i40_p145_a50.eps}}
% \caption{Model first moment maps for a flared disk (lower and upper cones) with a P.A.~of 
% 145\degree~and an inclination of 40\degree~(representative of the HD~100546 disk), 
% with an opening angle of 20\degree~(left), 35\degree~(middle), and 50\degree~(right).  
% The contours are in units of a single spectral resolution element (0.21 \kms).} 
% \label{figurea2}
% \end{figure*}
% 
% \begin{figure*}[]
% \subfigure{\includegraphics[width=0.33\textwidth]{./warp_mom1_r100_i025_p055.eps}}
% \subfigure{\includegraphics[width=0.33\textwidth]{./warp_mom1_r100_i050_p055.eps}}
% \subfigure{\includegraphics[width=0.33\textwidth]{./warp_mom1_r100_i075_p055.eps}}
% \subfigure{\includegraphics[width=0.33\textwidth]{./warp_mom1_r100_i080_p025.eps}}
% \subfigure{\includegraphics[width=0.33\textwidth]{./warp_mom1_r100_i080_p055.eps}}
% \subfigure{\includegraphics[width=0.33\textwidth]{./warp_mom1_r100_i080_p085.eps}}
% \caption{Model first moment maps for a warped disk with an outer disk P.A.~of 
% 145\degree, an inclination of 36\degree~, and an opening angle of 9\degree.  
% The top row of models have an inner warp with a fixed P.A.~of 55\degree~and 
% and inclination of 25\degree~(left), 50\degree~(middle), and 75\degree.  
% The bottom row of models have an inner warp with a fixed inclination of 80\degree~  
% (representative of that in the HD~100546 disk) and a position angle of 25\degree~
% (left), 55\degree~(middle), and 85\degree~(right).   
% The transition radius in all models is 100~au. 
% The contours are in units of a single spectral resolution element (0.21 \kms).} 
% \label{figurea3}
% \end{figure*}
\end{document}

% Figure and table environments

%\begin{deluxetable}{}
%\tablecaption{}
%\tablewidth{0pt}
%\tablehead{\colhead}
%\startdata
%\multicolumn{}{}{} \\
%\hline
% & \\
%\enddata
%\end{deluxetable}

%\begin{figure*}
%\includegraphics[width=\textwidth]{}
%\caption{}
%\label{figure1}
%\end{figure*}

%\begin{deluxetable*}{ccccc}
%\tablecaption{}
%\tablehead{\colhead{}}
%\startdata
% &  &  &  &  \\
%\enddata
%\end{deluxetable*}

% Appendix

%The line-of-sight positional offset of the top cone relative to the disk midplane is given by
%\begin{equation}
%\delta{_\mathrm{top}}(x,y,z) = \frac{r_\mathrm{mid}\tan{\phi}}{1-\tan{\phi}\tan{i}\cos{\theta}},
%\end{equation}
%with a similar expression for the bottom cone, 
%\begin{equation}
%\delta{_\mathrm{bot}}(x,y,z) = \frac{r_\mathrm{mid}\tan{\phi}}{1+\tan{\phi}\tan{i}\cos{\theta}},
%\end{equation}
%where $r_\mathrm{mid} = \sqrt{x^2+y^2+z_\mathrm{mid}^2}$, $z_\mathrm{mid} = y\tan{i}$, and 
%$i$ and $\phi$ have been defined previously. 
%The derivation of these expressions is given in the Appendix.  
%The radius of the top and bottom cone is then determined using,
%\begin{equation}
%r_\mathrm{top/bot}(x,y,z) = \sqrt{x^2 + y^2 + (z_\mathrm{mid} + \delta_\mathrm{top/bot})^2}.  
%\end{equation}
%The line-of-sight velocity can then be mapped back to the sky plane $(x',y')$ using 
%Equation \ref{velocity} and $r_\mathrm{top/bot}$ in the place of $r$ for the top and 
%bottom cones.

%Assuming that the minimum orbital period of the warp is the time span
%between the two APEX datasets ($\approx 6$ years), then the line 
%profile will cycle between blue-symmetric-red-symmetric-blue on a timescale of 
%$\approx 1.5$ years.  
%This means that, at the time that the ALMA data were taken, 
%the line profile was expected to be in its ``symmetric" phase, provided that 
%the earliest APEX data was taken in the middle of, or towards the end 
%of, the line profile's ``blue" phase.    
%If the above holds, this suggests the presence of a planetary companion 
%in a misaligned orbit at $\approx 3$~au.  
